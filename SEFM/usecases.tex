\section{Usecases}
\label{sec:usecases}

This section will give a summary of the usecases in
the ABCD online repository~\cite{usecase_repository}. At the
same time it will give a more detail presentation on
how session types are used in the implementation of the network application
usecase for a book-store, which is a standard example for introducing
session types. It will also focus on the implementation of a session type
Simple Mail Transfer Protocol, which is a standard network protocol.

\begin{figure}
	\begin{longtable}{|  l | p{3.2cm} | p{3cm} | p{2.5cm} | p{2.7cm} | }
		\hline
			&	\multicolumn{1}{c|}{Usecase}
							&	\multicolumn{1}{c|}{Domain}
														&	\multicolumn{1}{c|}{Technologies/Tools}
																					&	\multicolumn{1}{c|}{Description}
		\\
		\hline
		1.	&	Book Store	&	Business Application	&	Mungo, Session Java		&	Interaction Logic
		\\

		\hline
		2.	&	Chat Server	&	Network Application		&	Erlang					&	Application Logic
		\\

		\hline
		3.	&	Hyper-Text Transfer Protocol
							&	Network Protocol		&							&	Request/Response protocol
		\\

		\hline
		4.	&	Simple Mail Transfer Protocol
							&	Network Protocol		&	Mungo and StMungo, Session Java, LINKS
																					&	Stateful protocol
		\\
		\hline
		5.	&	Domain Name System
							&	Network Protocol / Service		&	Erlang
																					&	
		\\
		\hline
		6.	&	Concurrency	Problems
				\begin{enumerate}[label=$\bullet$]
					\item	Dinning Philosophers
					\item	Sleeping Barber
					\item	Cigarette Smokers
				\end{enumerate}
							&	Operating Systems		&	Python/Actors			&	Race conditions
		\\
		\hline
		7.	&	Lock		&	Operating Systems		&	Eventful Session Java	&	Race conditions
		\\

		\hline
		8.	&	Collection	&	Data Structures			&	Mungo					&	A stack client
		\\

		\hline
		9.	&	File Access	&	Data Structures and Algorithms
														&	Mungo					&	File Access client
		\\

		\hline
		10.	&	Concurrent Fibonacci
						&	Concurrent Algorithms		&	Mungo					&
		\\
		\hline
		11.	&	Network Topologies
				\begin{enumerate}[label=$\bullet$]
					\item	Ring
					\item	Butterfly
					\item	All to All
					\item	Stencil
					\item	Farm (Master-Worker)
					\item	Map Reduce
				\end{enumerate}
						&	Parallel Algorithms		&	Pabble and MPI				&	Parametric algorithms
		\\
		\hline
		12.	&	Concurrent Algorithms
				\begin{enumerate}[label=$\bullet$]
					\item	Peano Numbers
					\item	Add Server
				\end{enumerate}
						&	Concurrent Algorithms	&	Links, GV					&
		\\
		\hline
		13.	&	Memory Coherence
						&	Systems, Hardware		&	Mungo						&	
		\\
		\hline
	\end{longtable}
	\caption{Usecases in the ABCD online repository}
	\label{fig:usecases}
\end{figure}


A table summary for the usecases in the repository can be
found in~\figref{usecases}.

\subsection{Domain Classification of Usecases}

The usecases are presented following a wide range of
application domains, in order to demonstrate the fact that
session types can capture
a broad area of communication specifications. 
For the implementation of the usecases in the online
repository different technologies that integrate
session types in different programming paradigms were used.

%A taxonomy of domains are presented below with the main
%characteristic that each domain is using technologies from the
%previous domains in the taxonomy. This taxonomy also implies
%a stratification of the application of session types in different
%computation layers.
%%One of the goals of this paper is to present a diversity
%%of application of session types from different domains, as
%%part of our aim to demonstrate the robustness, functionality
%%and adaptability of session types. Here we present the particular
%%communication characteristics for every domain.

\begin{enumerate}
	\item	{\em Network Application/Business Logic}.
	Session types can be used to develop protocols for applications
	that run inside a network.
	A protocol given in a session type structure, apart from the
	specification of the communication of the application, will
	reveal a kind of business logic for the application.
	
	\item	{\em Network Protocols}.
	Session types can be used to describe standard and non-standard network protocols.
	Typically a standard network protocol should conform to
	an informal RFC (request for comments specification. Session types
	can present a network protocol formally its manipulation easier
	by both engineers and machines.
	Non-standard network protocols can also be developed.
	
	
	\item	{\em Systems/Applications}.
	A session type may be used to describe the communication
	specifics of an application that uses multiple resources
	inside a computing machine.
	
	\item	{\em Operating System}.
	Another domain where session types can be applied to
	is the description of the communication specifics
	of operating system algorithms and routines, that
	co-ordinate the usage of hardware resources.
	
	\item	{\em Data Structures and Algorithms}.
	The above layers are using data structures and algorithms.
	Session types can express the communication
	concurrent algorithms are using. Furthermore, session types
	can express the interaction with data structures.
	
	\item	{\em Hardware}.
	Hardware mechanisms complete the stratification of domains.
	The communication of hardware modules may also be expressed
	using session types.
	
	\item	\dk{\em Security}.
	Session types can also find applications in the security domain,
	which is a domain that supports all other domains in the above list.
\end{enumerate}
