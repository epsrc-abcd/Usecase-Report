% !TeX root = main.tex

\section{Use-Cases}
\label{sec:usecases}

This section summarises the use-cases in
the ABCD online repository~\cite{usecase_repository}. It also gives more detailed presentations of two particular use-cases: (1) the bookstore, which was introduced in Section~\ref{sec:session_types} and is a standard example of session types; (2) Simple Mail Transfer Protocol (SMTP), which is a standard network protocol. 

\newcommand{\JavaAPI}{Java API}
\newcommand{\Erlang}{Erlang}
\newcommand{\SJ}{SJ}
\newcommand{\SPython}{SPython}
\newcommand{\SScala}{SScala}
\newcommand{\TypeState}{Mungo}
\newcommand{\MPI}{MPI}
\newcommand{\Sill}{Sill}
\newcommand{\GV}{GV}
\newcommand{\Links}{Links}
\newcommand{\ESJ}{Eventful Session Java}

\newcommand{\rumi}[1]{$\mathbf{RN}$ {\color{red} #1} }


\begin{table}
\begin{tabular}{|l|l|l|l|l|}
\hline 
	No & Use Case Name & Source & Implementation & Remark \\


\hline \hline
 & \multicolumn{4}{|l|}{ \textbf{Network Protocols}} \\
\hline %\hline
	1 & SMTP & \cite{SMTP} & \JavaAPI, \TypeState, \Links & stateful protocol \\ 
	2 & HTTP & \cite{HTTP} & \JavaAPI & request-response \\
    3 & DNS & \cite{DNS} & \Erlang & \\
    4 & POP3 & \cite{POP3} & \TypeState & \rumi{Check with D.} \\ 
 %   4 & PAXOS & \cite{POP3} & \TypeState & \\
%    4 & OATH & \cite{OATH} & \JavaAPI \\
\hline \hline
 & \multicolumn{4}{|l|}{ \textbf{Application Protocols}} \\
\hline %\hline
    4 & Book Strore & \cite{BookStore} & \SJ, \Mungo, \JavaAPI  & interaction logic\\ 
	5 & Travel Agency & \cite{TravelAgency} & \SJ &\\
    6 & Chat Application & \cite{ChatApplication} & \Erlang &\\

\hline \hline
 & \multicolumn{4}{|l|}{ \textbf{Concurrency Patterns  and Algorithms}} \\
 \hline %\hline
	7 & Dining Philosophers & \cite{Savina} & \SPython  & synchronisation\\ 
	8 & Sleeping Barber & \cite{Savina} & \SPython, \SScala &  \\
    9 & Cigarette Smoker & \cite{Savina} & \SPython & race conditions\\
%    10 & Big & \cite{Savina} & \SPython & many-to-many interactions\\
    11 & Peano Numbers & \cite{} & \GV, \Links &\\
    12 & Add Server & \cite{} & \GV, \Links &\\
    13 & Concurrent Fibonacci & \cite{Fibonacci} & \SPython, \TypeState & \rumi{Check impl.} \\ 

\hline \hline
 & \multicolumn{4}{|l|}{ \textbf{Data Structures}} \\
\hline %\hline
	14 & Queue & \cite{Queue} & \Sill & \\ 
	15 & Collections & \cite{Stack} & \TypeState  & a stack client\\
    16 & File Access & \cite{FileAccess} & \TypeState & file access client\\
    
    
\hline \hline
 & \multicolumn{4}{|l|}{ \textbf{Network Topologies and Parallel Algorithms}} \\
\hline %\hline
	 17  &  Ring & \cite{BerkleyPar} & \MPI &\\
	 18 & Butterfly & \cite{BerkleyPar} & \MPI &\\
	 19 & All to All & \cite{BerkleyPar} & \MPI &\\
	 20 & Stencil & \cite{BerkleyPar} & \MPI &\\ 
	 21	& Farm (Master-Worker) & \cite{BerkleyPar} & \MPI &\\
	 22 & Map Reduce & \cite{BerkleyPar} & \MPI, \SPython &\\ 
\hline

\hline \hline
 & \multicolumn{4}{|l|}{\textbf{Operating Systems}} \\ 
\hline %\hline
 23  &  Memory Coherence & \cite{memoryCoherence} & \TypeState &\\
 24 & Lock & \cite{Lock} & \ESJ & race conditions\\
\hline
\end{tabular}
~\\
\caption{Overview of the use-case repository}
\label{table:use_cases_all}
\end{table}

%
%\begin{figure}[t]
%	\begin{longtable}{| c | p{3cm} | p{3cm} | p{2.5cm} | p{2.5cm} | }
%		\hline
%			&	\multicolumn{1}{c|}{Usecase}
%							&	\multicolumn{1}{c|}{Domain}
%														&	\multicolumn{1}{c|}{Technologies/Tools}
%																					&	\multicolumn{1}{c|}{Description}
%		\\
%		\hline
%		1.	&	Book Store	&	Business Application	&	Mungo,
%															Scribble API
%																					&	Interaction Logic
%		\\
%
%		\hline
%		2.	&	Chat Server	&	Network Application		&	Erlang					&	Application Logic
%		\\
%
%		\hline
%		3.	&	HTTP %Hyper-Text Transfer Protocol
%							&	Network Protocol		&	Scribble API			&	Request/Response %pattern
%		\\
%
%		\hline
%		4.	&	Simple Mail Transfer Protocol
%							&	Network Protocol		&	Mungo, Scribble API, LINKS
%																					&	Stateful protocol
%		\\
%		\hline
%		5.	&	DNS Server
%							&	Network Protocol 		&	Erlang
%																					&	
%		\\
%		\hline
%		6.	&	Concurrency
%				\parbox{3cm}{
%					$\bullet$	Din. Philosophers\\
%					$\bullet$	Sleeping Barber\\
%					$\bullet$	Cigarette Smokers
%				}
%				%\begin{enumerate}[label=$\bullet$]
%				%	\item	Dinning Philosophers
%				%	\item	Sleeping Barber
%				%	\item	Cigarette Smokers
%				%\end{enumerate}
%							&	Operating Systems		&	Python Actors			&	Race conditions
%		\\
%		\hline
%		7.	&	Lock		&	Operating Systems		&	Eventful SJ				&	Race conditions
%		\\
%
%		\hline
%		8.	&	Collection	&	Data Structures			&	Mungo					&	Stack client
%		\\
%
%		\hline
%		9.	&	File Access	&	Data Structures			&	Mungo					&	File Access client
%		\\
%
%		\hline
%		10.	&	Fibonacci
%						&	Parallel Algorithms			&	Mungo					&
%		\\
%		\hline
%		11.	&	Network Topologies
%				\parbox{3cm}{
%					$\bullet$	Ring\\
%					$\bullet$	Ring\\
%					$\bullet$	Butterfly\\
%					$\bullet$	All to All\\
%					$\bullet$	Stencil\\
%					$\bullet$	Master-Worker\\
%					$\bullet$	Map Reduce
%				}
%%				\begin{enumerate}[label=$\bullet$]
%%					\item	Ring
%%					\item	Butterfly
%%					\item	All to All
%%					\item	Stencil
%%					\item	Farm (Master-Worker)
%%					\item	Map Reduce
%%				\end{enumerate}
%						&	Parallel Algorithms		&	Pabble and MPI				&	Parametrised algorithms
%		\\
%		\hline
%		12.	&	Functional Algor.
%				\parbox{3cm}{
%					$\bullet$	Peano Numbers\\
%					$\bullet$	Add Server
%				}
%
%%				\begin{enumerate}[label=$\bullet$]
%%					\item	Peano Numbers
%%					\item	Add Server
%%				\end{enumerate}
%						&	Parallel Algorithms		&	Links, GV					&
%		\\
%		\hline
%		13.	&	Memory Coherence
%						&	Systems, Hardware		&	Mungo						&	
%		\\
%		\hline
%	\end{longtable}
%	\caption{Usecases in the ABCD online repository}
%	\label{fig:usecases}
%\end{figure}


Table~\ref{table:use_cases_all} lists the name of each use-case, its original source, and the language(s) and/or tool(s) that have been used to implement it. Full source code, 
running examples and and detailed descriptiona of all of the use-cases can be found in the repository \cite{usecase_repository}. 
The use-cases are organised into application domains and are intended to be representative examples for each domain.

\subsection{Technologies based on Session Types}


Several technologies and tools based on session types have been used to implement the use-cases that we describe in Section~\ref{sec:usecases}.
%
\begin{enumerate}
	\item	Session Java~\cite{HU07TYPE-SAFE} is an extension of Java with binary session types for communication channels, supported by a runtime library. The compiler statically checks session types. 

	\item	Eventful Session Java~\cite{event} supports asynchronous, event-driven programming, using session types to track progress through individual sessions.

	\item	Multiparty Session C~\cite{NYH12} supports programming in C with the MPI library. The compiler statically checks session types. 

	\item	SPY~\cite{DBLP:conf/rv/NeykovaYH13} uses session types as the basis for runtime monitoring of communication protocols in Python.

	\item	Two technologies apply session types to the actors paradigm:
			\begin{itemize}
				\item	An implementation that uses SPY to
						monitor python threads that simulate actors~\cite{DBLP:conf/coordination/NeykovaY14}.
				\item	A similar approach on monitoring session types is also used for the
						Erlang programming language. %, which is a programming language for the Actors model.
\sg{Is this Erlang system one of ours? Simon Fowler's MSc}
			\end{itemize}

	\item	Pabble~\cite{DBLP:conf/pdp/NgY14} extends Scribble to express
			communication structures that are parameterised by the number of roles.
			Pabble protocols are implemented and typechecked for the C+MPI framework.

	\item	Mungo~\cite{mungo} is a tool that integrates session types into the object-oriented
			programming paradigm through the notion of typestate. Communication
			operations on a channel are accessed via a state-dependent interface.
			%Mungo uses a protocol description language for capturing typestate.
			The Scribble to Mungo (StMungo) tool transforms Scribble
			protocols into Mungo interfaces.


	\item	GV:			A functional programming implementation of binary session types. \sg{Refer to Phil's paper.}
	\item	The web programming language Links \cite{} includes static typechecking of protocols, using the linear logic interpretation of session types \cite{}. 
	\item	SILL:		Functional programming based on the linear dual intuitionistic interpretation of session types. \sg{Separate this from the repository --- say something elsewhere.}

	\item	A state pattern implementation as a library. Uses Scribble to automatically
			create an API for state pattern programming and uses runtime checks to
			cope with linearity requirements.

\end{enumerate}

\subsection{Application Domains Covered by the Use-Cases}

The use-cases are drawn from a wide range of
application domains, in order to demonstrate that
session types can capture
a broad area of communication specifications. 
%For the implementation of the usecases in the online
%repository different technologies that integrate
%session types in different programming paradigms were used.

%A taxonomy of domains are presented below with the main
%characteristic that each domain is using technologies from the
%previous domains in the taxonomy. This taxonomy also implies
%a stratification of the application of session types in different
%computation layers.
%%One of the goals of this paper is to present a diversity
%%of application of session types from different domains, as
%%part of our aim to demonstrate the robustness, functionality
%%and adaptability of session types. Here we present the particular
%%communication characteristics for every domain.

\begin{enumerate}
	\item	\emph{Network Application / Business Logic}.
			The first two applications in Table~\ref{table:use_cases_all}
			come from the domain of network applications.

			The first is the online bookstore,
			which was discussed
			in Section~\ref{sec:session_types}. It is a standard use-case demonstrating
			the basic features of session types. The communication protocol represents the execution logic of the application. The implementation of this use-case follows a top-down design approach, in which the global protocol is projected to local protocols. There are several versions of the bookstore, using different implementation languages: Session Java 

			The bookstore usecase is implemented following a
			communication protocol that embeds an execution logic
			for the application. This is a practical indication
			that session types can be used to describe the
			behavioural logic of an application.
			The current development the usecase follows a
			top-down design approach, where the projection
			of the global protocol is used by different
			implementing technologies.
			The book-store is implemented in Session Java and Mungo.
			\dk{need LINKS and the hybrid implementation}

			\mybf{- Book-store: Standard example, Communication Interaction shows business logic}

			\mybf{- Chat-Server: Implement a server/client architecture, design an application protocol for the servers,
				client implementations should conform to the protocol, Erlang monitoring implementation cite S. Fowler's Master Thesis.}
		

%			is implemented in different technologies

%	Session types can be used to develop protocols for applications
%	that run inside a network.
%	A protocol given in a session type structure, apart from the
%	specification of the communication of the application, will
%	reveal a kind of business logic for the application.
	
	\item	{\em Network Protocols}.
	Session types can be used to describe standard and non-standard network protocols.
	Typically a standard network protocol should conform to
	an informal RFC (request for comments specification. Session types
	can present a network protocol formally its manipulation easier
	by both engineers and machines.
	Non-standard network protocols can also be developed.

	\mybf{- HTTP: Request response standard RFC protocol, stateless,
			assume different types for each request/response header (not just a text header)
			request-response sequences can give a further structure
			of interaction, it would be nice if we have a client to stream tree-structured data.
			automatically generated API from the Scribble tool-chain} 

	\mybf{- SMTP: Stateful standard RFC protocol. Expresses request/respond, 
				Expresses specific data-structures streaming such as the data-structure
				of an email,
				need for translate between RFC text format to/from message/payload types
				Complex state makes implementation difficult - session types
				remove a burden from the developer because they automatically
				verify that the state of the code follows the state of the protocol. }

	\mybf{- Domain Name System: }
	
	
	\item	{\em Systems/Applications}.
	A session type may be used to describe the communication
	specifics of an application that uses multiple resources
	inside a computing machine.

	\mybf{- Do we have a system to describe }
	
	\item	{\em Operating System}.
	Another domain where session types can be applied to
	is the description of the communication specifics
	of operating system algorithms and routines, that
	co-ordinate the usage of hardware resources.

	\mybf{- Locks: Basic OS structure basic for achieving resource utilisation. Its
		usage implies race conditions that in turn imply asynchronous
		and reactive communication. The implementation of locks in the
		eventful session Java shows that session types can express deadlocks.}

	\mybf{- Concurrency Algorithms: Classic concurrency problems expressed in Session types.
		Implementing in the python monitored actors framework.}
        \rumi{
% We have implemented well-known synchronisation problems.   
%We address a classical concurrency problems. 
Classic concurrency problems require correct coordination among multiple components to avoid starvation and deadlock. 
Preserving the causalities between the interactions is challenging since the communication often involves complex patterns, combining long sequence of interactions with recursive behaviour and nested choice branches. 
Often precise message sequence should be followed. Moreover, it is not obvious if the components can be safely composed. Sending the wrong message type, sending to the wrong role or not sending in the correct message sequence may lead to deadlocks, errors which initial cause is hard to be identified or wrong computation results.
Modelling the interactions between the components with session types ensures correct synchronisation and prevents deadlocks, unexpected termination and orphan messages.  \\

Sleeping barber: A barber is waiting for customers to cut their hair. When there are no customers the barber sleeps. When a customer arrives, he wakes the barber or sits in one of the waiting chairs in the waiting room. If all chairs are occupied, the customer leaves. Customers repeatedly visit the barber shop until they get a
haircut. The key element of the solution is to make sure that whenever a customer or a barber checks the state of the waiting room, they always see a valid state. The problem can be implemented using an additional Selector role that decides which is the next customer. \\

Dining Philosophers: In this use case N philosophers are competing to eat while sitting
around a round table. There is one fork between each two philosophers and a philosopher needs
two forks to eat. The challenge is to avoid deadlock, where no one can eat, because everyone
is possessing exactly one fork. The problem can be implemented with additional role Arbitrator. \\

Cigarette Smoker: This problem involves N smokers and one arbiter. The arbiter puts resources
on the table and smokers pick them. A smoker needs to collect k resources to start smoking. The
challenge is to avoid deadlock by disallowing a competition between smokers from picking up
resources. This is done by delegating the control to the arbiter, who decides (in a random manner)
which smoker to send the resource to. The session types for this use case combines round-robin
pattern, sending a random smoker a message to smoke, with multicast, iterating
through all the smokers notifying them to exit). \\

\mybf{Concurrent Fibonacci: Parallel algorithms require threads that communicate. Session
		types can use describe the necessary underlying communication between threads that implement
		parallel algorithms.}
%three components, room, barber,and customer. A barber waits for customers in his shop, sleeping when there is nobody to serve. When a customer enters in the shop, he goes through a waiting room with n chairs: if all chairs are taken, he leaves; otherwise, he sits. If the barber is sleeping, he wakes up, serves all sitting customers (one a time), and sleeps again when nobody is waiting. We model this scenario with three components: the customer, the shop and the barber.

% and is the key to achieving deadlock free communication.
%To guarantee a valid synchronisation, precise message sequence should be followed. 
%The communication structure these protocols requires long sequence of interactions, combined with recursive behaviour and multiple nested choice branches.  

%While the use cases so far demonstrated succinct recursion and request-reply patterns, interleaved together, this one demonstrates a communication structure with long sequence of interactions, where preserving the causalities is the key to achieving deadlock free communication. The use of session types prevents potential deadlock, unexpected termination and orphan messages.    		
		}
	
	\item	{\em Data Structures and Algorithms}.
	The above layers are using data structures and algorithms.
	Session types can express the communication
	concurrent algorithms are using. Furthermore, session types
	can express the interaction with data structures.

	\mybf{- Collection: A stack client protocol. Used to control the put, get access
		to a collection structure - when the stack is empty there is not get.
		Session types describe the logic/properties that a data structure can have.}

	\mybf{- File Access: Similarly used to control the access on a file. Cannot
		read from a file if the file is not open first and if the file is empty.
		The protocols ends by closing the file. Again access to resources
		can be described using session types.}

	\mybf{- Concurrent Fibonacci: Parallel algorithms require threads that communicate. Session
		types can use describe the necessary underlying communication between threads that implement
		parallel algorithms.}

	\mybf{- Network Topologies: Topologies are scalable and thus parametrised. Pabble is
		an extension to Scribble that allows us to describe and cope with parametrised protocols.}


	\item	{\em Hardware}.
	Hardware mechanisms complete the stratification of domains.
	The communication of hardware modules may also be expressed
	using session types.

	\mybf{- Memory Coherency: Hardware components communicate, here we have two memories
		that need to be consistent with each other on a hardware level. Session types can
		describe the hardware (signals/messages) interaction between hardware components.}

	\item	\mybf{\em Security}.
	Session types can also find applications in the security domain,
	which is a domain that supports all other domains in the above list.

	\mybf{- There are some security protocols in the SILL repository. We need to
			be careful in the description because it is a reference to other people.}
\end{enumerate}

\subsection{The Simple Mail Transfer Protocol}

This section gives a more detailed presentation of the SMTP
usecase. SMTP is considered an important implementation for
session types since it incorporates challenges that in principle
software engineers face.

The SMTP is a protocol that, in contrast with the HTTP, requires handling
communication through a number of distinct states (\mybf{see state machine diagram below}),
which adds a degree of complexity to the implementation.
With the help of session types and the methods and procedures
session types support developers can reduce the effort
needed to handle the complexity.

Furthermore, SMTP introduces a structure on an email,
which is composed by subject, main part and attachments
as well as meta-data for the email. Streaming such 
tree-structured data types over the network is a task that
can be rigorously handled and ensured by session types,
due to its structured natured. Transforming
network data-structures (e.g.~text headers)
to/from session types structures is an important
procedure to understand when applying session types. 

%In this report we describe the implementation of
%an application layer communication protocol.

SMTP is an application layer protocol.
{SMTP} is an internet standard
electronic mail transfer protocol which typically runs over a TCP
(Transmission Control Protocol) connection.
It was first defined in RFC
821~\cite{SMTP-rfc} and later extended in RFC 5321~\cite{ESMTP-rfc},
which is the version we consider here.


An SMTP interaction consists of an exchange of text-based {commands}
between the client and the server.
For example, the client sends the
\lstinline|EHLO| command to identify itself and open the connection with
the server.
%
The commands \lstinline|MAIL FROM| : $<$address$>$ and \lstinline|RCPT TO| : $<$address$>$
specify the e-mail address of the sender and the
receiver of the e-mail.
%
The \lstinline|DATA| command allows the client to specify the text of
the e-mail. The \lstinline|QUIT| command is used to terminate the
session and close the connection. The responses from the server have the
following format: three digits followed by an optional dash ``-'', such
as \lstinline|250-|, and then some text, like OK. The server might reply
to \lstinline|EHLO| with \lstinline|250| $<$text$>$ or to
\lstinline|MAIL FROM| or \lstinline|RCPT TO| with \lstinline|250| OK.


The implementation of the protocol implements an interaction between
a Simple Mail Transfer Protocol (SMTP) client and a \lstinline|gmail| server.


The global protocol for the SMTP is defined in Scribble, 
based on Hu's work~\cite{HuR:smtp}.

\begin{lstlisting}[numbers=left]
global protocol SMTP(role S, role C) {
 // Global interaction between server and client.
  _220(String) from S to C;
   choice at C {
    ehlo(String) from C to S;
    rec X {
      choice at S {
        _250dash(String) from S to C;
         continue X;
      } or {
       _250(String) from S to C;
        choice at C {
         ...
         rec X1 {
           ...
           choice at S {
             _250dash(String) from S to C;
             continue X1;
          } or {
            250(String) from S to C;
            ...
            rec Z1 {
              ...
              data(String) to S;
              ...
              rec Z3 {
                choice at C {
                  subject(String) from C to S;
                  continue Z3;
                } or {
                  dataline(String) form C to S;
                  continue Z3;
                } or {
                  atad(String) from C to S;
                  _250(String) from S to C;
                  continue Z1; }
                }
              }   ...  
           }  
           ... 
        }
      }
    }
  }
 } or  { quit(String) from C to S; }
}
\end{lstlisting}



The above Scribble protocol formally specifies
the description in \secref{sec: smtp_description}.
The global protocol \lstinline|SMTP| specifies
an interaction between two roles being the client ant the server of the protocol.
The protocol is then projected to the
individual communication specifications of each 
of the participant roles.

%%
The global protocol is used to project the communication
behaviour of roles \lstinline|C| and \lstinline|S|.
The focus is on the projection of the client
since the interest is on interacting with a real server.

This part of SMTP describes a loop (\lstinline|rec Z1|) where the client chooses
among the messages \lstinline|SUBJECT|, to send the subject,
\lstinline|DATALINE|, to send a line of text, or \lstinline|ATAD| to terminate the e-mail by sending a dot.

\begin{lstlisting}[numbers=left]
local protocol SMTP_C(role S, self C) {
  _220(String) from S;
  choice at C{
    ehlo(String) to S;
    ...
    rec Z1 {
      ...
      data(String) to S;
      ...
      rec Z3 {
        choice at C {
          subject(String) to S;
          continue Z3;
        } or {
          dataline(String) to S;
          continue Z3;
        } or {
          atad(String) to S;
          _250(String) from S;
          continue Z1; }}}
      ...
  } or { quit(String) to S; }
}
\end{lstlisting}
%


The local type can be represented as a state
machine:

%\begin{statemachine}[node distance=6cm, scale=0.5]
%	\node[initial,state]		(S0)							{\lstinline|State0|};
%	\node[state]				(S1)[right of=S0]				{\lstinline|State1|};
%	\node[state]				(S2)[below =2cm of S1]			{\lstinline|State2|};
%	\node[state]				(S3)[left of=S2]				{\lstinline|State3|};
%	\node[state]				(S29)[below = 2cm of S3]		{\lstinline|State29|};
%	\node[state]				(S30)[right of= S29]		{\lstinline|State30|};
%	\node[state]				(S32)[left = 3cm of S29]		{\lstinline|State32|};
%	\node[state]				(S31)[below = 2cm of S29]		{\lstinline|State31|};
%
%%	\node[state]				(S39)[right = 9cm of S31]		{\lstinline|State39|};
%	\node[state]				(S39)[above = 2cm of S32]		{\lstinline|State39|};
%%	\node[state,accepting]		(END)[below =2cm of S39]		{\lstinline|END|};
%	\node[state,accepting]		(END)[above =2cm of S39]		{\lstinline|END|};
%	\node[state]				(SX)[below = 2cm of S31]		{\lstinline|Other States|};
%
%	\path	(S0)		edgenode		{\lstinline|String receive_220StringFromS()|}	(S1)
%			(S1)		edgenode[left]	{\lstinline|void send_EHLOToS()|}	(S2)
%						edgenode[below= 0.4cm]	{\lstinline|void send_QUITToS()|}	(S39)
%			(S2)		edgenode		{\lstinline|void send_ehloStringToS(String)|}	(S3)
%			(S3)		edgenode		{\lstinline|...|}	(S29)
%			(S29)		edgenode 		{\lstinline|void send_SUBJECTToS()|}	(S30)
%						edgenode		{\lstinline|void send_DATALINEToS()|}	(S31)
%						edgenode[above]	{\lstinline|void send_ATADToS()|}	(S32)
%			(S32)		edgenode		{\lstinline|...|}	(SX)
%			(S31)		edgenode		{\lstinline|...|}	(SX)
%			(S30)		edgenode		{\lstinline|...|}	(SX)
%			(S39)		edgenode[right]	{\lstinline|void send_quitStringToS(String)|}	(END);
%\end{statemachine}
%%%


\begin{center}
\scalebox{0.8}{
\begin{statemachine}[node distance=4.5cm]
	\node[initial,state]		(S0)							{\lstinline|S0|};
	\node[state]				(S1)[right of=S0]				{\lstinline|S1|};
	\node[state]				(S2)[below =0.5cm of S1]		{\lstinline|S2|};
	\node[state]				(S3)[left of=S2]				{\lstinline|S3|};
	\node[state]				(S29)[below = 1.5cm of S3]		{\lstinline|S29|};
	\node[state]				(S32)[right of = S29]				{\lstinline|S32|};
	\node[state]				(S30)[above = 0.5cm of S32]				{\lstinline|S30|};
	\node[state]				(S31)[below = 0.5cm of S32]		{\lstinline|S31|};
	
	\node[state]				(S39)[right of = S1]			{\lstinline|S39|};
	\node[state,accepting]		(END)[below =0.5cm of S39]		{\lstinline|END|};
	\node[state]				(SX)[below = 1.5cm of END]		{\lstinline|Other|};
	
	\path
	(S0)		edgenode			{\smallcode{receive\_220StringFromS()}}	(S1)
	(S1)		edgenode[left]		{\smallcode{send\_EHLOToS()}}	(S2)
				edgenode			{\smallcode{send\_QUITToS()}}	(S39)
	(S2)		edgenode			{\smallcode{send\_ehloStringToS(String)}}	(S3)
	(S3)		edgenode[left]		{\smallcode{...}}	(S29)
	(S29)		edgenode[right]		{\smallcode{send\_SUBJECTToS()}}	(S30)
				edgenode			{\smallcode{send\_DATALINEToS()}}	(S31)
				edgenode			{\smallcode{send\_ATADToS()}}	(S32)
	(S32)		edgenode			{\smallcode{...}}	(SX)
	(S31)		edgenode[right]		{\smallcode{...}}	(SX)
	(S30)		edgenode			{\smallcode{...}}	(SX)
	(S39)		edgenode[left]		{\smallcode{send\_quitStringToS(String)}}	(END);
\end{statemachine}
}
\end{center}
%%

