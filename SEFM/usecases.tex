% !TeX root = main.tex

\section{Use-Cases}
\label{sec:usecases}

This section summarises the ABCD on-line
repository~\cite{usecase_repository} use-cases.
%that feature session types.
The description focuses on the
solutions session types offers in the different 
use-case domains. During the discussion
the outline details of a selection of individual
use-case is given. The discussion goes
into more depth on two particular canonical
use-cases:
(1) the bookstore, a standard session types example introduced
in Section~\ref{sec:session_types};
% commonly found in the literature;
and
(2) Simple Mail Transfer Protocol (SMTP),
a standardised Internet application protocol.

Table~\ref{table:use_cases_all} lists the name of each use-case,
its original source, and the language(s) and tool(s) that have been
used to implement it. Full source code, running examples and detailed
description of all of the use-cases can be found in~\cite{usecase_repository}. 
Use-cases are organised into application domains and are intended
to be representative examples for each domain. Some of the use-cases
were presented independently in the bibliography.

\subsection{Application Domains Covered by the Use-Cases}

The use-cases are drawn from a wide range of
application domains, in order to demonstrate that
session types can capture a broad area of communication
specifications and a wide variety of patterns.
On  one hand, applying session types to diversity of use-cases
demonstrates a range of features and extensions in the tools. 
%
On the other  hand, each domain offers appreciation
of session types from a different perspective.

%\cut{We explain the specific challenges tackled by session types
%when applied to a specific domain.
%%In this section we demonstrate that depending on the domain
%%the properties are important to a various degree. 
%%Each domain has its own challenges and although session
%%types provide multiple guarantees, the motivation to apply
%%session types to a specific domain is slightly different.
%Communication safety guarantees are essential when
%programming in an error-prone environment
%(as in High Performance Computing).
%Session fidelity guarantee is essential when protocols
%are specified in a semi-formal format
%(as in Internet Application Protocols).
%Ensuring progress is challenging when a complex
%coordination of multiple parties is required (as is Web Services)
%and compositionality is not apparent
%(as in Classic Concurrency Problems).
%\dk{do we need this long paragraph}}

\newcommand{\JavaAPI}{Java API}
\newcommand{\Erlang}{Erlang}
\newcommand{\SJ}{SJ}
\newcommand{\SPython}{SPython}
\newcommand{\SScala}{SScala}
\newcommand{\TypeState}{Mungo}
\newcommand{\MPI}{MPI}
\newcommand{\Sill}{Sill}
\newcommand{\GV}{GV}
\newcommand{\Links}{Links}
\newcommand{\ESJ}{Eventful Session Java}

\newcommand{\rumi}[1]{$\mathbf{RN}$ {\color{red} #1} }


\begin{table}
\begin{tabular}{|l|l|l|l|l|}
\hline 
	No & Use Case Name & Source & Implementation & Remark \\


\hline \hline
 & \multicolumn{4}{|l|}{ \textbf{Network Protocols}} \\
\hline %\hline
	1 & SMTP & \cite{SMTP} & \JavaAPI, \TypeState, \Links & stateful protocol \\ 
	2 & HTTP & \cite{HTTP} & \JavaAPI & request-response \\
    3 & DNS & \cite{DNS} & \Erlang & \\
    4 & POP3 & \cite{POP3} & \TypeState & \rumi{Check with D.} \\ 
 %   4 & PAXOS & \cite{POP3} & \TypeState & \\
%    4 & OATH & \cite{OATH} & \JavaAPI \\
\hline \hline
 & \multicolumn{4}{|l|}{ \textbf{Application Protocols}} \\
\hline %\hline
    4 & Book Strore & \cite{BookStore} & \SJ, \Mungo, \JavaAPI  & interaction logic\\ 
	5 & Travel Agency & \cite{TravelAgency} & \SJ &\\
    6 & Chat Application & \cite{ChatApplication} & \Erlang &\\

\hline \hline
 & \multicolumn{4}{|l|}{ \textbf{Concurrency Patterns  and Algorithms}} \\
 \hline %\hline
	7 & Dining Philosophers & \cite{Savina} & \SPython  & synchronisation\\ 
	8 & Sleeping Barber & \cite{Savina} & \SPython, \SScala &  \\
    9 & Cigarette Smoker & \cite{Savina} & \SPython & race conditions\\
%    10 & Big & \cite{Savina} & \SPython & many-to-many interactions\\
    11 & Peano Numbers & \cite{} & \GV, \Links &\\
    12 & Add Server & \cite{} & \GV, \Links &\\
    13 & Concurrent Fibonacci & \cite{Fibonacci} & \SPython, \TypeState & \rumi{Check impl.} \\ 

\hline \hline
 & \multicolumn{4}{|l|}{ \textbf{Data Structures}} \\
\hline %\hline
	14 & Queue & \cite{Queue} & \Sill & \\ 
	15 & Collections & \cite{Stack} & \TypeState  & a stack client\\
    16 & File Access & \cite{FileAccess} & \TypeState & file access client\\
    
    
\hline \hline
 & \multicolumn{4}{|l|}{ \textbf{Network Topologies and Parallel Algorithms}} \\
\hline %\hline
	 17  &  Ring & \cite{BerkleyPar} & \MPI &\\
	 18 & Butterfly & \cite{BerkleyPar} & \MPI &\\
	 19 & All to All & \cite{BerkleyPar} & \MPI &\\
	 20 & Stencil & \cite{BerkleyPar} & \MPI &\\ 
	 21	& Farm (Master-Worker) & \cite{BerkleyPar} & \MPI &\\
	 22 & Map Reduce & \cite{BerkleyPar} & \MPI, \SPython &\\ 
\hline

\hline \hline
 & \multicolumn{4}{|l|}{\textbf{Operating Systems}} \\ 
\hline %\hline
 23  &  Memory Coherence & \cite{memoryCoherence} & \TypeState &\\
 24 & Lock & \cite{Lock} & \ESJ & race conditions\\
\hline
\end{tabular}
~\\
\caption{Overview of the use-case repository}
\label{table:use_cases_all}
\end{table}

%
%\begin{figure}[t]
%	\begin{longtable}{| c | p{3cm} | p{3cm} | p{2.5cm} | p{2.5cm} | }
%		\hline
%			&	\multicolumn{1}{c|}{Usecase}
%							&	\multicolumn{1}{c|}{Domain}
%														&	\multicolumn{1}{c|}{Technologies/Tools}
%																					&	\multicolumn{1}{c|}{Description}
%		\\
%		\hline
%		1.	&	Book Store	&	Business Application	&	Mungo,
%															Scribble API
%																					&	Interaction Logic
%		\\
%
%		\hline
%		2.	&	Chat Server	&	Network Application		&	Erlang					&	Application Logic
%		\\
%
%		\hline
%		3.	&	HTTP %Hyper-Text Transfer Protocol
%							&	Network Protocol		&	Scribble API			&	Request/Response %pattern
%		\\
%
%		\hline
%		4.	&	Simple Mail Transfer Protocol
%							&	Network Protocol		&	Mungo, Scribble API, LINKS
%																					&	Stateful protocol
%		\\
%		\hline
%		5.	&	DNS Server
%							&	Network Protocol 		&	Erlang
%																					&	
%		\\
%		\hline
%		6.	&	Concurrency
%				\parbox{3cm}{
%					$\bullet$	Din. Philosophers\\
%					$\bullet$	Sleeping Barber\\
%					$\bullet$	Cigarette Smokers
%				}
%				%\begin{enumerate}[label=$\bullet$]
%				%	\item	Dinning Philosophers
%				%	\item	Sleeping Barber
%				%	\item	Cigarette Smokers
%				%\end{enumerate}
%							&	Operating Systems		&	Python Actors			&	Race conditions
%		\\
%		\hline
%		7.	&	Lock		&	Operating Systems		&	Eventful SJ				&	Race conditions
%		\\
%
%		\hline
%		8.	&	Collection	&	Data Structures			&	Mungo					&	Stack client
%		\\
%
%		\hline
%		9.	&	File Access	&	Data Structures			&	Mungo					&	File Access client
%		\\
%
%		\hline
%		10.	&	Fibonacci
%						&	Parallel Algorithms			&	Mungo					&
%		\\
%		\hline
%		11.	&	Network Topologies
%				\parbox{3cm}{
%					$\bullet$	Ring\\
%					$\bullet$	Ring\\
%					$\bullet$	Butterfly\\
%					$\bullet$	All to All\\
%					$\bullet$	Stencil\\
%					$\bullet$	Master-Worker\\
%					$\bullet$	Map Reduce
%				}
%%				\begin{enumerate}[label=$\bullet$]
%%					\item	Ring
%%					\item	Butterfly
%%					\item	All to All
%%					\item	Stencil
%%					\item	Farm (Master-Worker)
%%					\item	Map Reduce
%%				\end{enumerate}
%						&	Parallel Algorithms		&	Pabble and MPI				&	Parametrised algorithms
%		\\
%		\hline
%		12.	&	Functional Algor.
%				\parbox{3cm}{
%					$\bullet$	Peano Numbers\\
%					$\bullet$	Add Server
%				}
%
%%				\begin{enumerate}[label=$\bullet$]
%%					\item	Peano Numbers
%%					\item	Add Server
%%				\end{enumerate}
%						&	Parallel Algorithms		&	Links, GV					&
%		\\
%		\hline
%		13.	&	Memory Coherence
%						&	Systems, Hardware		&	Mungo						&	
%		\\
%		\hline
%	\end{longtable}
%	\caption{Usecases in the ABCD online repository}
%	\label{fig:usecases}
%\end{figure}


%As we have already explained, session types ensure multiple properties. Here we review session types properties wrt. thir importance when implementing use cases in different domains. 
%We have identified domains that offer a different viewpoint and appreciation of the properties. 
%This is essential when dealing with concurrent algorithms where. 
%Describing the specification in terms of a formal description language gives clarity and reduces the programming errors: 
%(1) Internet application protocols require thorough understanding of long and tedious semi-formal specifications. 
%(2) communication safety. In cases when the communication mismatches and synchronisation of messages is a common source of errors 
%(3) when use cases involve large number of participants and coordination between the participants becomes challenging. Large number of participants. 
 
%Covering a wide variety of patterns which allows comparison of features available in the session-based frameworks being evaluated.
%For the implementation of the usecases in the online
%repository different technologies that integrate
%session types in different programming paradigms were used.

%A taxonomy of domains are presented below with the main
%characteristic that each domain is using technologies from the
%previous domains in the taxonomy. This taxonomy also implies
%a stratification of the application of session types in different
%computation layers.
%%One of the goals of this paper is to present a diversity
%%of application of session types from different domains, as
%%part of our aim to demonstrate the robustness, functionality
%%and adaptability of session types. Here we present the particular
%%communication characteristics for every domain.

%%%%%%%%%%%%%%%%%%%%%%%%%%%%%%%%%%%
%%% Web Service Applications
%%%%%%%%%%%%%%%%%%%%%%%%%%%%%%%%%%%

\myparagraph{Web Service Application}
%Internet Application Protocols are low level protocols,
%as such they often specify a protocol between two
%parties (a client and a server).
The aim of the use cases presented in this domain is to
demonstrate the usage of multi-party protocols, as descriptions of
the communication behaviour of network components.
In web services, applications make an extensive use
of communications among multiple components and services
through a standardised format and agreed-upon protocol structure.
%Business transactions using web services
%are often termed \textit{business protocols} because each of them
%obey an agreed-upon protocol structure.
The protocol stipulates the business logic to
be implemented by the application.

To show the applicability of session types to
this domain we implemented use cases
(Book Store, Travel Agency) designed by the WS-CDL Working Group \cite{W3C},
intended to represent general concepts common to many applications of web services

The book-store is a canonical example for demonstrating 
business logic interaction.
Its global and  local protocol for \BuyerOne
can be found in~\figref{scribble_bs}. Two
Buyer roles, \BuyerOne and \BuyerTwo, interact with a \Seller
role to buy a book. \BuyerOne requests a quote for a book.
The \Seller replies with a price and then \BuyerOne asks
\BuyerTwo for a contribution. \BuyerTwo then makes a choice:
i) \BuyerTwo can agree to contribute and informs both \BuyerOne
and the \Seller, with the price transferring interactions taking
place afterwards; ii) \BuyerTwo does not agree and ends the interaction
informing both \BuyerOne and \Seller.
%
It is implemented in Mungo using the automatic code generation of
StMungo. The effort of the programmer was focused on filling the
decision logic for the values exchanged on the message interactions.
\dk{put the other implementations}

%\cut{The web-services set of use cases is a practical indication that session
%types can be used to describe the behavioural logic of an application.
%The current development of the use-cases follows a top-down
%design approach, where the projection of the global protocol is
%used by different implementing technologies.}

%\ray{...multiparty, ``richer'' data types (XML, SOAP, etc), delegation...}

%%%%%%%%%%%%%%%%%%%%%%%%%%%%%%%%%%%
% Internet Application Protocols
%%%%%%%%%%%%%%%%%%%%%%%%%%%%%%%%%%%

\myparagraph{Internet Application protocols}
This set of use cases is the first step towards
more realistic (real-world) applications.
The motivation to apply session types verification
techniques to Internet Application Protocols is three-fold.
%: (1) programs should be interropable (2) it may involve untrusted components (3) error-prone specifications.

First, an Internet Application Protocol should conform to
a RFC (request for comments specification), which is an error prone
procedure due to:
(1) the semi-formal nature of the protocol description; and
(2) the rich interaction structures described
(as observed in SMTP and POP3 use-cases).
Session protocols %is a formal language and is
are expressive enough to capture the communication
patterns required for the popular use-cases in this domain.  
Second, the aim of the specification is to enable
interactions between heterogeneous systems.
The software development method allows for component
verification in different technologies
%The Scribble development methodology can be used to
%verify components implemented in different languages. 
Third, implementations often involve untrusted components.
Using session-based runtime monitors both incoming and outgoing
messages of a component are verified, thus protecting the
implemented components from interactions with untrusted parties.
For example, the DNS implementation in the repository
is implemented in monitored Erlang but it can interact with any
(none session-based) server implementations.
%\cut{Monitoring ensures correct communication behaviour with
%	untrusted or erroneous components.}

Another widely used internet protocol implemented in the repository is HTTP.
Despite, the fact that HTTP is a basic request-response protocol,
it has a number of different commands that require attention.
HTTP's global specification can be defined in many different ways:
e.g.~a request/response pattern with large message types. The approach
taken here is to construct a request (resp., response) header as
a stream of simpler messages passed that construct the overall
request (resp., response) header. The implementation in the online
repository first describes the HTTP in Scribble and then automatically
generates a Java API.
%
The SMTP is also implemented as a use-case in the repository.
A detailed description of SMTP can be found in \secref{smtp}.


%\cut{\ray{...binary (thin abstraction layer over TCP session),
%can have ``rich'' interaction structure (e.g. SMTP, POP3 conversations),
%simple data types (ASCII text)...}}

%\rumi{For each domain explain (1) What are the challenges when implementing the use cases in that particular domain?
%(2) How Session Types overcome these challenges (currently this is not well explained).}

%%%%%%%%%%%%%%%%%%%%%%%%%%%%%%%%%%%%
% Classic Concurrency Problems
%%%%%%%%%%%%%%%%%%%%%%%%%%%%%%%%%%%%

\myparagraph{Classic Concurrency Problems}
Coordination of processes is a challenging problem
when implementing concurrent systems.
A study in \cite{ActorCoordinationStudy} points out that
``the property of no shared space and asynchronous communication
can make implementing coordination protocols harder and
providing a language for coordinating protocols is needed''.
The problem is best exemplified when implementing
classic concurrency problems. They require correct coordination
among multiple components to avoid starvation and deadlock.
Preserving the causalities between interactions is
challenging since the communication often involves complex patterns,
that combine long sequence of interactions with recursive behaviour
and nested choice branches. Moreover, it is not obvious if the components can be safely composed.
Sending a wrong message type, or sending to the wrong role,
or not sending in the correct message sequence may lead to deadlocks,
errors which initial cause is hard to be identified or wrong computation results.

%The paper reveals that the coordination of processes is a challenging problem when implementing, and especially when scaling up concurrent systems.

Modelling interactions between components with session types ensures
correct synchronisation and prevents deadlocks,
unexpected termination and orphan messages as demonstrated by the implemented use cases. The repository includes correct Scribble protocols and verified endpoint implementations in Python for classic problems of dining philosophers, sleeping barber, and
cigarette smoker. The sleeping barber problem is also implemented in Session Scala. The problem is modelled with three components: the customer, the shop and the barber, using session types to formalise their expected interactions such that the customer and/or the barber are always synchronised on the number of waiting customers in the shop. The implementation shows how multiple concurrent sessions (one per customer) can be handled by single-threaded programs (the shop and the barber). It also shows how to address the problem with session delegation, by leveraging higher-order session types. 

Both Scala and Python implementations rely on a custom session types libraries and do not require language extensions. The Scala library provides static verification, while the Python library uses runtime monitors.  

%Expresses specific data-structures streaming such as the data-structure of an email,need for translate between RFC text format to/from message/payload types.

%Benefits of ST usage: interoperability, interaction with non-verified components. Motivation for dynamic monitoring,
%as well as static verification. They are low-level protocols and thus inherently binary.

%%%%%%%%%%%%%%%%%%%%%%%%%%%%%%%%%%%%%%%%%%%%%
% Parallel Algorithms
%%%%%%%%%%%%%%%%%%%%%%%%%%%%%%%%%%%%%%%%%%%%%

\myparagraph{Parallel Algorithms}
Issues with High Performance Computing (HPC)
programmers productivity and programs correctness
have long been recognised.
A survey among MPI (the de facto standard programming model for HPC)
programmers~\cite{MPIErrors} ranks
\textit{communication mismatches} as the most common errors.
Communication mismatches include:
(1) send/receive inconsistency caused by an error in
program execution flow; and
(2) send/receive inconsistency caused by incorrect
sender and/or receiver specified for a message.
These are precisely the type of errors prevented with
session types verification.

The Jacobi and Monte Carlo algorithms are classic parallel algorithms
that require session types to ensure no errors in their
concurrent implementation.
%
In addition, concurrent Fibonacci is an
example that shows the underlying communication
pattern needed by a parallel algorithm.
The algorithm assumes two nodes that store
consecutive Fibonacci numbers. An exchange
in numbers allows for the calculation of the
next Fibonacci number.


%Parallel algorithms require threads that communicate. Session types can use describe the necessary underlying communication between threads that implement parallel algorithms.
%\rumi{The highlights of this domain are: (1) computation and communication are often separated, (2) topologies are scalable and thus parametrised and (3) topologies are reusable, comprising limited set of patterns.}

%X suggests that the goal of computer science should be making parallel computing productive, efficient, correct, portable, and scalable.

%%%%%%%%%%%%%%%%%%%%%%%%%%%%%%%%%%%%%%%
% Network Topologies
%%%%%%%%%%%%%%%%%%%%%%%%%%%%%%%%%%%%%%%

\myparagraph{Network Topologies}
The previously presented use-cases in the repository
involved a fixed number of interacting parties.
Network topologies, however, often express indexed and grouped processes interaction patterns with unknown number of participants. To overcome this problem, the protocol description
language Pabble is used, which is an extension
of the Scribble language featuring parametrised
roles.
% Pattern-based structured parallel programming motivates the need for specifications of flexible topologies.
Pabble can express all structured patterns in the
HPC (High Performance Computing) Dwarf benchmark suit~\cite{BerkleyPar} (i.e. algorithmic methods that capture common pattern of communication and computation), 
%In addition to specific parallel algorithms we also evaluate
%session types on a set of patterns for parallel programming.
%which is a popular benchmark suit that captures
%common parametrised patterns of communication.
%The Seven Dwarfs, constitute equivalence classes where
%membership in a class is defined by similarity in computation
%and data movement.
%The dwarfs are specified at a high level of abstraction to
%allow reasoning about their behaviour across a broad
%range of applications.
%Dwarf suit captures common patterns of communication and computation. %The Seven Dwarfs, constitute equivalence classes where membership in a class is defined by similarity in computation and data movement.
%The dwarfs are specified at a high level of abstraction to allow reasoning about their behavior across a broad range of applications. %Programs that are members of a particular class can be implemented differently and the underlying numerical methods may change over time, but the claim is that the underlying patterns have persisted through generations of changes and will remain important into the future.
%The dwarfs present a method for capturing the common requirements of classes of applications while being reasonably divorced from individual implementations.

%To exemplify the advantages of pattern programming –
%common parallel patterns are collected and stored in our protocol repository, and
%they are maintained separately from the implementations so new parallel applications
%can be constructed by writing only the computations logic. 

We provide an implementation for each pattern (topology) from the table using a tool for automatic MPI code generation
from reusable Pabble specifications. Note that one parallel algorithm can be implemented using multiple network topologies. Thus, this use-cases set is a generalisation of the parallel algorithm use-cases. For example, the repository contains implementations of the Monte-Carlo $\pi$ simulation with two different patterns, scatter-gather and master-worker (the former uses collective operations and all processes are involved in the main calculation, whereas with the latter
workers are coordinated by a central master process by P2P communication that does not perform the main calculation).
The benchmark results in ~\cite{NCY2015} shown that the practise
of code generation from Pabble specifications improves developers
productivity in terms of development and debugging efforts.

%It improves productivity of parallel application development with reusable patterns.

\myparagraph{Data Structures}
The last domain of use-cases describes the behaviour of
clients accessing data-structures that are implemented
concurrently. The main feature here are:
(1) a session protocol can be used to restrict
a client to a data-structure in using specific operations; and
(2) often data-structures require an ordering in the
operations, where such ordering is described using session protocols.

For example, the collection use-case features a client
that has access to a stack data-structure. The client
is allowed to arbitrary push elements to the stack, but
it may need to check if the stack is empty before it
pops an element from the stack.
A second example is the file access use-case, where:
(i) a file needs to be opened before it can be processed;
(ii) there is an empty file check before a read;
(iii) the close file operation ends the protocol; and
(iv) the client is not allowed any other file operations then
the operations in (i), (ii), and (iii).
Both use-cases are implemented in Mungo.

%\myparagraph{Other Domains}
%
%{\em Operating System}.
%	Another domain where session types can be applied to
%	is the description of the communication specifics
%	of operating system algorithms and routines, that
%	co-ordinate the usage of hardware resources.
%
%	\mybf{- Locks: Basic OS structure basic for achieving resource utilisation. Its
%		usage implies race conditions that in turn imply asynchronous
%		and reactive communication. The implementation of locks in the
%		eventful session Java shows that session types can express deadlocks.}
%
%
%{\em Data Structures and Algorithms}.
%	The above layers are using data structures and algorithms.
%	Session types can express the communication
%	concurrent algorithms are using. Furthermore, session types
%	can express the interaction with data structures.
%
%	\mybf{- Collection: A stack client protocol. Used to control the put, get access
%		to a collection structure - when the stack is empty there is not get.
%		Session types describe the logic/properties that a data structure can have.}
%
%	\mybf{- File Access: Similarly used to control the access on a file. Cannot
%		read from a file if the file is not open first and if the file is empty.
%		The protocols ends by closing the file. Again access to resources
%		can be described using session types.}
%{\em Hardware}.
%	Hardware mechanisms complete the stratification of domains.
%	The communication of hardware modules may also be expressed
%	using session types.
%
%	\mybf{- Memory Coherency: Hardware components communicate, here we have two memories
%		that need to be consistent with each other on a hardware level. Session types can
%		describe the hardware (signals/messages) interaction between hardware components.}
%
%{\em Security}.
%	Session types can also find applications in the security domain,
%	which is a domain that supports all other domains in the above list.
%
%	\mybf{- There are some security protocols in the SILL repository. We need to
%			be careful in the description because it is a reference to other people.}



%\subsection{``Use-Case highlights''}
%
%\paragraph{Book-Store}
%
%
%\mybf{put more}
%
%\paragraph{SMTP}
%
%\mybf{put SMTP here}
%
%\paragraph{HTTP.}
%
%
%\mybf{
%\paragraph{Travel Agency.}
%...W3C CDL working group Web services use case, initially specified as binary sessions
%between multiparty roles (not a single multiparty session), features delegation...

%\paragraph{Parallel algorithms.}
%...shared memory as transport, integrating linearity for session typing into
%language as general linearity/aliasing control for shared memory message passing optimisation...

%\paragraph{..OOI python use cass...}
%...TODO: add use cases? (e.g., negotiation, resource access control)... ...python: run-time monitoring motivation...
%}

%\paragraph{Collection Data-Structure and File Access}


%\paragraph{Concurrent Fibonacci}
%
%
%
%\paragraph{others?}

%...

