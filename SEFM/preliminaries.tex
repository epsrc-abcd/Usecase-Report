\section{Preliminaries}
\label{sec:preliminaries}

In this section the basic notions and terms that are used throughout
the paper, are defined and discussed.

\subsection{Session Types: Notions and Terminology}

%Session types can be seen as types that describe behaviour.
Typically, a type is broadly understood
as a meta-information concept, used describe a class of a data.
Structures of types are used to describe data structures.
When types are structured as input/output requirements they
are used to describe functionality.
The combination of functionality and data structures give rise
to the concepts of classes and interfaces.% of objects.
%
A session type extends %the above approaches of
the notion of a type to capture the communication behaviour in concurrent systems.
% definition of session types
{\em A session type is a type structure that describes a communication
behaviour in terms of a series of send and receive %message passing
interactions between a set of concurrent entities.}

A session type may also be seen as a
formal {\em specification} of a communication {\em protocol} among concurrent {\em roles}.
The syntax for session types is defined on {\em send} and {\em receive},
and {\em select} and {\em branch} operators.
Sending and receiving data gives rise to the notion of {\em message passing}.
In a session type messages are described via their types.
The case where a session typed value is passed as a message is called {\em session delegation}.
The interaction of select and branch defines the {\em choice} interaction.
Choice requires for a role to use a value called {\em label}
to select a communication behaviour out of a set of behaviours, called {\em branch},
offered by another role.

Session types assume a {\em global} protocol that describes the
communication interaction of all the roles inside a concurrent system.
The perspective of a global protocol through a single role
is called {\em local} protocol. The local protocol describes
the communication interaction of a single role with all other roles
in the system.
%Session types also assume a {\em local} protocol that describes the
%communication interaction of a single role with all other roles within the system.

The relation between a global protocol and the
local protocols of its roles, is expressed
through the {\em projection} procedure;
% called {\em projection}.
%Projection to a role
%is a procedure that can be done
%can be done automatically;
given the global protocol and a role, projection 
returns as a local protocol only the
interactions of the global protocol that are concerned with
the role.

The reverse procedure from projection is called {\em synthesis},
where a set of local protocols are composed together in a global
protocol.

Below the term system is used to describe the whole concurrent
system and the term \dk{program} is used for communication module
inside the system.

Typically, a protocol is first expressed globally and then
projected locally on the protocol roles.
Local specifications are then implemented by the corresponding
\dk{programs}.
The term {\em session fidelity} is used to express
the conformance of a \dk{program}, to the local protocol of the
corresponding role. Session fidelity can be checked mechanically,
through techniques like type-checking and monitoring.

If all the roles of a protocol are implemented by a
system and session fidelity is ensured,
then a number of communication properties can be guaranteed
for the communication behaviour of the system:
%
\begin{enumerate}[label=$\bullet$]
	\item	Every execution state of the system has {\em safe} communication behaviour:
	%	
	\begin{itemize}
		\item	{\em Communication operator matching}: Every send (resp., select) operation has a corresponding receive (resp., branch) operation.
		\item	{\em Message type matching}: The type of every message being send matches the type of the message expected to be receive.
		\item	{\em Deadlock-freedom}: Consequently, every message send will be eventually received.
	\end{itemize}
	
	\item	Every state of the protocol will {\em progress} to a safe state.
\end{enumerate}

\subsection{Session Types as Protocols: The Scribble Tool-chain}

This section presents the Scribble~\cite{scribble} tool-chain
as the core tool for the application of session types.
This section also uses Scribble to clarify the terminology
developed in the previous section.

The basic module of the Scribble tool-chain is the
Scribble protocol description language,
which is a syntax that expresses session type specifications.
The design of the Scribble language draws directly from
the principles of session types.

\begin{figure}[t]
\begin{lstlisting}
  global protocol Bookstore(role Buyer1, role Buyer2, role Seller) {
    book(title) from Buyer1 to Seller;
    book(price) from Seller to Buyer1;
    quote(contribution) from Buyer1 to Buyer2;
    choice at Buyer2 {
      agree() fom Buyer2 to Buyer1, Seller;
      transfer(money) from Buyer1 to Seller;
      transfer(money) from Buyer2 to Seller;
    } or {
      quit() from Buyer2 to Buyer1, Seller;
    }
  }
\end{lstlisting}
\caption{Scribble: Global protocol for the Bookstore usecase}
\label{fig:scribble_bs}
\end{figure}


In \figref{scribble_bs}, uses the Scribble protocol for the
Bookstore usecase in the online repository~\cite{usecase_repository},
to present the Scribble syntax.
Line 1 defines the 
of global protocol \lstinline|Bookstore| between
roles \lstinline|Buyer1|, \lstinline|Buyer2| and \lstinline|Seller|.
%The communication structure is expressed 
%as a sequence of {\em endpoint} to endpoint message passing
%interactions between the roles of the protocol.
%Messages are described in terms of a message type,
%where  a label is used to annotate a list
%of types.
Line 2 gives an example of the simplest endpoint-to-endpoint
message passing interaction,
where a message with
label \lstinline|book| and type \lstinline|title| (notation \lstinline|book(title)|)
is send from (keyword \lstinline|from|) role \lstinline|Buyer1| to (keyword \lstinline|to|)
role \lstinline{Seller}.
In Scribble messages are described as a type structure
consisted of a label and a list of message types.


Line 5 of the protocol clarifies the choice interaction, where
role \lstinline|Buyer2| selects a choice from two possible outcomes,
expressed with the labels \lstinline|agree| or \lstinline|quit|. In 
both cases roles \lstinline|Buyer1| and \lstinline|Seller| offer
alternative branches.



\dk{describe the recursion construct}


\begin{figure}[t]
\begin{lstlisting}
local protocol Bookstore at Buyer1(role Buyer1, role Buyer2, role Seller) {
  book(title) to Seller;
  book(price) from Seller;
  quote(contribution) to Buyer2;
  choice at Buyer2 { agree() fom Buyer2; transfer(money) to Seller; }
  or               { quit() from Buyer2; }
}
\end{lstlisting}
	\caption{Local Protocol for Role Buyer1}
\end{figure}



%We use the term {\em local} protocol to describe
%the communication structure at the level of a single participant.
%As session types is a type discipline with the requirement to
%be enforced on a program, we may sometimes refer to a session type
%protocol as communication {\em specification}.

\begin{comment}
For example Scribble code:
%
\begin{lstlisting}
  msg(int) from A to B;
\end{lstlisting}
%
describes the global protocol between role \lstinline|A| and \lstinline|B|,
where participant \lstinline|A| sends message of type \lstinline|msg(int)|
to participant \lstinline|B|. From the local perspective of participant
\lstinline|A| the protocol would be:
%
\begin{lstlisting}
  msg(int) to B;
\end{lstlisting}
%
where it describes the sending of message \lstinline|msg(int)|
to participant \lstinline|B|.
\end{comment}



\subsection{Domain Classification of Usecases}

The paper presents a diversity of usecase scenarios of
session types from different application domains, in order
to demonstrate the fact that session types can capture
a broad area of communication specifications.
A taxonomy of domains are presented below with the main
characteristic that each domain is using technologies from the
previous domains in the taxonomy. This taxonomy also implies
a stratification of the application of session types in different
computation layers.
%One of the goals of this paper is to present a diversity
%of application of session types from different domains, as
%part of our aim to demonstrate the robustness, functionality
%and adaptability of session types. Here we present the particular
%communication characteristics for every domain.

\begin{enumerate}
	\item	{\em Network Application/Business Logic}.
			Session types can be used to develop protocols for applications
			that run inside a network.
			A protocol given in a session type structure, apart from the
			specification of the communication of the application, will
			reveal a kind of business logic for the application.

	\item	{\em Network Protocols}.
			Session types can be used to describe standard and non-standard network protocols.
			Typically a standard network protocol should conform to
			an informal RFC (request for comments specification. Session types
			can present a network protocol formally its manipulation easier
			by both engineers and machines.
			Non-standard network protocols can also be developed.


	\item	{\em Systems/Applications}.
			A session type may be used to describe the communication
			specifics of an application that uses multiple resources
			inside a computing machine.

	\item	{\em Operating System}.
			Another domain where session types can be applied to
			is the description of the communication specifics
			of operating system algorithms and routines, that
			co-ordinate the usage of hardware resources.

	\item	{\em Data Structures and Algorithms}.
			The above layers are using data structures and algorithms.
			Session types can express the communication
			concurrent algorithms are using. Furthermore, session types
			can express the interaction with data structures.

	\item	{\em Hardware}.
			Hardware mechanisms complete the stratification of domains.
			The communication of hardware modules may also be expressed
			using session types.

	\item	\dk{\em Security}.
			Session types can also find applications in the security domain,
			which is a domain that supports all other domains in the above list.
\end{enumerate}


\subsection{Usecase Classification on Technologies used}
\dk{Just presentation and references}

The technologies used to implement the demonstrated usecases are presented
below:
%
\begin{enumerate}
	\item	Session Java~\cite{HU07TYPE-SAFE} is the first tool
			developed as a tool with a library that support binary session
			types. The tool requires compiler support to perform session
			type-check.

	\item	The Eventful Session Java~\cite{event} is an extension
			of session types to support asynchronous, event-driven programming.

	\item	Multiparty Session C~\cite{NYH12}
			is the first tool that implemented a
			multiparty session type library for MPI.
			The tool requires compiler support to perform session
			type-check.


	\item	Session fidelity can also be ensured using runtime monitoring.
			Modules for monitoring were implemented in python and are presented in~\cite{DBLP:conf/rv/NeykovaYH13}.

	\item	The actors paradigm
			\begin{itemize}
				\item	Implemented in python using the monitor module
				\item	Erlang
			\end{itemize}
	\item	Pabble - MPI: parametric Scribble used to express parametric algorithms and topologies.
	\item	Mungo and StMungo:	based on the OO paradigm a tool that integrates session types and the 
								typestate theory in Java. StMungo transforms Scribble into Mungo
								specifications.
	\item	GV:			A functional programming implementation of binary session types.
	\item	LINKS:		Functional programming based on the linear logic interpretation of session types.
	\item	SILL:		Functional programming based on the linear dual intuitionistic interpretation of session types.

	\item	A state pattern implementation as a library. Uses Scribble to automatically
			create an API for state pattern programming and uses runtime checks to
			cope with linearity requirements.

\end{enumerate}

\subsection{Fine/coarse grain representation}
\dk{this should go as future work}

{\em Concurrency patterns} e.g. network topology, client/server, race condition, etc. (maybe I am writing nonsense here)

