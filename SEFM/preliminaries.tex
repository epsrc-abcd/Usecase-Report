\section{Preliminaries}

Before we proceed with the demonstration of the main thesis
of this paper, we identify the basic notions and set the basic
terminology that we are going to use throughout the paper.

\subsection{Session Types as Protocols: The Scribble Language}

In this section we use Scribble~\cite{scribble} as the core
medium to develop a terminology for presenting session types.
The terminology we use is defined in a non mathematical terms,
as it was the case until now. Instead we use general terms
that are used by a wider computer science community.
Scribble is a protocol description language with its design
drawn directly from the principles of session types.

In \figref{scribble_bs}, we use the Scribble protocol for the
Bookstore usecase in the online repository~\cite{usecase_repository},
to present the Scribble syntax and the terminology for session types.


\begin{figure}[t]
\begin{lstlisting}
  global protocol Bookstore(role Buyer1, role Buyer2, role Seller) {
    book(title) from Buyer1 to Seller;
    book(price) from Seller to Buyer1;
    quote(contribution) from Buyer1 to Buyer2;
    choice at Buyer2 {
      agree() fom Buyer2 to Buyer1, Seller;
      transfer(money) from Buyer1 to Seller;
      transfer(money) from Buyer2 to Seller;
    } or {
      quit() from Buyer2 to Buyer1, Seller;
    }
  }
\end{lstlisting}
\caption{Scribble: Global protocol for the Bookstore usecase}
\end{figure}

Session types are used to express communication
{\em protocols} in terms of {\em structured communication}.
A {\em global} communication structure is expressed 
as a sequence of {\em endpoint} to endpoint {\em message passing}
interactions between the {\em roles} of the protocol.
We use the term {\em local} protocol to describe
the communication structure at the level of a single participant.
As session types is a type discipline with the requirement to
be enforced on a program, we may sometimes refer to a session type
protocol as communication {\em specification}.

For example Scribble code:
%
\begin{lstlisting}
  msg(int) from A to B;
\end{lstlisting}
%
describes the global protocol between role \lstinline|A| and \lstinline|B|,
where participant \lstinline|A| sends message of type \lstinline|msg(int)|
to participant \lstinline|B|. From the local perspective of participant
\lstinline|A| the protocol would be:
%
\begin{lstlisting}
  msg(int) to B;
\end{lstlisting}
%
where it describes the sending of message \lstinline|msg(int)|
to participant \lstinline|B|.

There is a tight relation between a global protocol and the
local protocols of its roles, that is expressed
through an automated procedure called {\em projection}.
Typically, a protocol is first expressed globally and then
projected locally on the protocol roles. Roles
correspond to the computation processes of the scenario
that are expected to implement their corresponding local type.
We use the term {\em session fidelity} to express the fact
that a protocol participant implements its corresponding local type.

In terms of properties, session types can ensure, it is proven
that session types ensure

send/receive {\em duality}
{\em deadlock freedom}
{\em type soundness} on the message being passed


\subsection{Session Type Disciplines and Technologies}

\dk{Just presentation and references}

{\em Binary} session types - present it just for academic reasons

{\em Multiparty session types} - overcomes limitation and subsumes binary

{\em Parametric session types} - number of participants as parameter - network topologies

{\em Dynamic session types} - ...

{\em Eventful session types} - ...

{\em Functional session types} - Gay Vasconcelos paper, linear logic interpretation

{\em Typestate and session types} - ...

{\em Actors} - Python, erlanf

{\em Monitors}

\subsection{Classification of Usecases}

{\em domain taxonomy}

{\em technologies} used

{\em  fine/coarse grain} representation of a scenario

{\em Concurrency patterns} e.g. network topology, client/server, race condition, etc. (maybe I am writing nonsense here)

