%%%%%%%%%%%% SMTP

\subsection{The Simple Mail Transfer Protocol}
\label{sec:smtp}

The SMTP is an internet standard for electronic
mail transfer, that follows the RFC 5321\footnote{Extended Simple Mail Transfer Protocol, RFC 5321,
\url{https://tools.ietf.org/html/rfc5321}}.
%It originally followed
%RFC 821~\cite{SMTP-rfc} and was later extended
%in RFC 5321~\cite{ESMTP-rfc}, which is the version consider here.
SMTP is an application layer protocol, that typically runs on
a TCP/IP connection.

\myparagraph{Motivation:}
The session typed implementation of the SMTP has to offer
a number of motivations for using session types for software
development:
(1) it demonstrates how {\em natural} it is
to describe of a widely used standard protocol in session types; and
(2) it is a rather stateful protocol (\appref{smtp_global}),
i.e.~ it requires implementation of a sequence of communication states.
The complexity of states adds to the effort of
developing SMTP. With the use of session typed
support that effort can be reduced.
%Thirdly, the SMTP implies streaming a tree
%data-structure that corresponds to the structure
%of an email.
%It is thus, shown how session types can describe
%the streaming of tree data-structures.


%This section gives a more detailed presentation of the SMTP
%usecase. SMTP is considered an important implementation for
%session types since it incorporates challenges that in principle
%software engineers face.
%
%The SMTP is a protocol that, in contrast with the HTTP, requires handling
%communication through a number of distinct states ,
%which adds a degree of complexity to the implementation.
%With the help of session types and the methods and procedures
%session types support developers can reduce the effort
%needed to handle the complexity.
%
%Furthermore, SMTP introduces a structure on an email,
%which is composed by subject, main part and attachments
%as well as meta-data for the email. Streaming such
%tree-structured data types over the network is a task that
%can be rigorously handled and ensured by session types,
%due to its structured natured. Transforming
%network data-structures (e.g.~text headers)
%to/from session types structures is an important
%procedure to understand when applying session types.

%%In this report we describe the implementation of
%%an application layer communication protocol.

\myparagraph{Session Type Specification}
The implementation of the SMTP starts with
the definition a Scribble global protocol~\cite{HuR:smtp}
that describes the interaction between a client and the server.
Part of the global protocol can be found in \appref{smtp_global}.
The global specification involves a number of interactions
that include nested recursion and choices, thus the SMTP
is a rather state-full protocol.
%
An SMTP interaction is an exchange of text-based {commands}
between a client and the server.
%
For example, in line 4 the client may send the
\lstinline|EHLO| command to identify itself and open a server connection.
Responses from the server have follow  the format:
three digits followed by an optional dash ``-'', and then some text,
e.g.~in line 8 the server might reply to \lstinline|EHLO| with \lstinline|_250 <text>| \mybf{check here}.
%
%Commands \lstinline|MAIL FROM: <address>| and \lstinline|RCPT TO: <address>|
%specify the e-mail address of the sender and the receiver of the e-mail.
%
%The \lstinline|DATA| command allows the client to specify the text of
%the e-mail. The \lstinline|QUIT| command is used to terminate the
%session and close the connection.
%Responses from the server have the following format:
%three digits followed by an optional dash ``-'', such
%as \lstinline|250-|, and then some text, e.g.~OK.
%The server might reply to \lstinline|EHLO| with \lstinline|250 <text>|
%or to \lstinline|MAIL FROM| or \lstinline|RCPT TO| with \lstinline|250 OK|.
%
Lines 18-23 there is a loop that allows the client to construct
and stream a tree data structure that contains the email information
to the server. Specifically, in line 19 the client streams the subject of the
email to the server and then performs iteration to stream email data lines,
as in line 20. The streaming finishes in line 21 and 22 when the
\lstinline|atad| message is sent and the server replies to the client.

The protocol is then projected to the local specifications
of the roles server and client. Part of the local protocol
of the client can be found in \appref{smtp_local}.
Local protocols can be represented as state machines. The state
machine for the client protocol can be found in \appref{state_machine}.
%%
%The global protocol is used to project the communication
%behaviour of roles \lstinline|C| and \lstinline|S|.
%The focus is on the projection of the client
%since the interest is on interacting with a real server.

%This part of SMTP describes a loop (\lstinline|rec Z1|) where the client chooses
%among the messages \lstinline|SUBJECT|, to send the subject,
%\lstinline|DATALINE|, to send a line of text, or \lstinline|ATAD| to terminate the e-mail by sending a dot.

%\begin{lstlisting}[numbers=left]
local protocol SMTP_C(role S, self C) {
  _220(String) from S;
  choice at C{
    ehlo(String) to S;
    ...
    rec Z1 {
      ...
      data(String) to S;
      ...
      rec Z3 {
        choice at C {
          subject(String) to S;
          continue Z3;
        } or {
          dataline(String) to S;
          continue Z3;
        } or {
          atad(String) to S;
          _250(String) from S;
          continue Z1; }}}
      ...
  } or { quit(String) to S; }
}
\end{lstlisting}
%



%%\begin{statemachine}[node distance=6cm, scale=0.5]
%	\node[initial,state]		(S0)							{\lstinline|State0|};
%	\node[state]				(S1)[right of=S0]				{\lstinline|State1|};
%	\node[state]				(S2)[below =2cm of S1]			{\lstinline|State2|};
%	\node[state]				(S3)[left of=S2]				{\lstinline|State3|};
%	\node[state]				(S29)[below = 2cm of S3]		{\lstinline|State29|};
%	\node[state]				(S30)[right of= S29]		{\lstinline|State30|};
%	\node[state]				(S32)[left = 3cm of S29]		{\lstinline|State32|};
%	\node[state]				(S31)[below = 2cm of S29]		{\lstinline|State31|};
%
%%	\node[state]				(S39)[right = 9cm of S31]		{\lstinline|State39|};
%	\node[state]				(S39)[above = 2cm of S32]		{\lstinline|State39|};
%%	\node[state,accepting]		(END)[below =2cm of S39]		{\lstinline|END|};
%	\node[state,accepting]		(END)[above =2cm of S39]		{\lstinline|END|};
%	\node[state]				(SX)[below = 2cm of S31]		{\lstinline|Other States|};
%
%	\path	(S0)		edgenode		{\lstinline|String receive_220StringFromS()|}	(S1)
%			(S1)		edgenode[left]	{\lstinline|void send_EHLOToS()|}	(S2)
%						edgenode[below= 0.4cm]	{\lstinline|void send_QUITToS()|}	(S39)
%			(S2)		edgenode		{\lstinline|void send_ehloStringToS(String)|}	(S3)
%			(S3)		edgenode		{\lstinline|...|}	(S29)
%			(S29)		edgenode 		{\lstinline|void send_SUBJECTToS()|}	(S30)
%						edgenode		{\lstinline|void send_DATALINEToS()|}	(S31)
%						edgenode[above]	{\lstinline|void send_ATADToS()|}	(S32)
%			(S32)		edgenode		{\lstinline|...|}	(SX)
%			(S31)		edgenode		{\lstinline|...|}	(SX)
%			(S30)		edgenode		{\lstinline|...|}	(SX)
%			(S39)		edgenode[right]	{\lstinline|void send_quitStringToS(String)|}	(END);
%\end{statemachine}
%%%


\begin{center}
\scalebox{0.8}{
\begin{statemachine}[node distance=4.5cm]
	\node[initial,state]		(S0)							{\lstinline|S0|};
	\node[state]				(S1)[right of=S0]				{\lstinline|S1|};
	\node[state]				(S2)[below =0.5cm of S1]		{\lstinline|S2|};
	\node[state]				(S3)[left of=S2]				{\lstinline|S3|};
	\node[state]				(S29)[below = 1.5cm of S3]		{\lstinline|S29|};
	\node[state]				(S32)[right of = S29]				{\lstinline|S32|};
	\node[state]				(S30)[above = 0.5cm of S32]				{\lstinline|S30|};
	\node[state]				(S31)[below = 0.5cm of S32]		{\lstinline|S31|};
	
	\node[state]				(S39)[right of = S1]			{\lstinline|S39|};
	\node[state,accepting]		(END)[below =0.5cm of S39]		{\lstinline|END|};
	\node[state]				(SX)[below = 1.5cm of END]		{\lstinline|Other|};
	
	\path
	(S0)		edgenode			{\smallcode{receive\_220StringFromS()}}	(S1)
	(S1)		edgenode[left]		{\smallcode{send\_EHLOToS()}}	(S2)
				edgenode			{\smallcode{send\_QUITToS()}}	(S39)
	(S2)		edgenode			{\smallcode{send\_ehloStringToS(String)}}	(S3)
	(S3)		edgenode[left]		{\smallcode{...}}	(S29)
	(S29)		edgenode[right]		{\smallcode{send\_SUBJECTToS()}}	(S30)
				edgenode			{\smallcode{send\_DATALINEToS()}}	(S31)
				edgenode			{\smallcode{send\_ATADToS()}}	(S32)
	(S32)		edgenode			{\smallcode{...}}	(SX)
	(S31)		edgenode[right]		{\smallcode{...}}	(SX)
	(S30)		edgenode			{\smallcode{...}}	(SX)
	(S39)		edgenode[left]		{\smallcode{send\_quitStringToS(String)}}	(END);
\end{statemachine}
}
\end{center}
%%


\myparagraph{Implementation}
The above session protocol drives the implementations:
(1) of an SMTP client in Mungo via StMungo;
(2) the SMTP client and server in LINKS; and
(3) the generation of a Java API that can be used
for implementing SMTP servers and clients.
In all the above technologies,
the implementation of the SMTP protocol was
aided by automated procedures that can handle
the communication structure of a program. Furthermore,
the implementation need not to be validated for
communication correctness, since if the session protocol
is correct and session fidelity is ensured then
the implementation is correct.

The Mungo implementation of the protocol
implements an interaction between an SMTP client
and the \lstinline|gmail| server. It uses the StMungo
tool to translate the client local protocols into Mungo
typestate specifications. StMungo also generates template
code that implements the communication functionality of the client.
The final implementation is type-checked by the Mungo tool against
the generated typestate specification.

LINKS implements SMTP server and client by first translating
the local protocols into LINK channel types. The programmer
then develops the client (resp., server) as a session typed
channel, used to exchange information with the server (resp., client).

Finally, Scribble offers the possibility of generating a Java API;
each protocol state corresponds to a generated class, with each
class implementing as methods the communication operations available
in each state. Method calls return an object corresponding to the next state.
A runtime check ensures that each object/state is used only once.
The Java API can guide the developer in order to implement a variety
of SMTP clients and servers.
