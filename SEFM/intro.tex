\section{Introduction}





The session types discipline emerged during the last two decades,
as a framework for enforcing a number of {\em good} and desirable properties
in concurrent programs.
Since their original inception in the
the seminal paper by Honda et.~al.~\cite{Honda.Vasconcelos.Kubo},
there has been a broad research effort to study the properties of
session types and at the same time apply session types to different
frameworks and paradigms. This subsequently lead to the development
of tools and technologies that use session types as part of the 
software development process.

On the one hand, research on session types is strongly motivated and
driven by the need to demonstrate that session types principles can
to capture arbitrary patterns in concurrency,
i.e.~patterns that are not restricted to particular cases.
On the other hand, research on session types%, and especially motivation and conclusion,
is often presented in isolation with respect to other results.
Furthermore, research on session types is strongly connected
with a detailed theoretical study.
Subsequently, a wider audience would need to study a series
of partial results and at the same time filter out any
unnecessary information in order to understand session types.

% Huge volume of info
% theoretical results
% 

In this paper we identify the problem of accessibility of session
types. We draw motivation by the need to demonstrate session
types in terms that are easier to understand by a wider audience.
The method we use is the method of demonstrating our practical
experience in session types that is reflected by a set of
usecase scenarios~\cite{online_repository} that we have implemented as part of 
%In this paper we draw motivation from the need to present a comprehensive
%survey of session types, using the means of demonstration. A broad spectrum
%of technologies are currently used to research on session types and as
%a result there are many implementations of different usecases available
%to the research community. As part of
the ABCD project~\cite{abcd_webpage}\footnote{From Data Types to Session Types: A Basis for Concurrency and Distribution}.

More concretely we aim to:
\begin{itemize}
	\item	Demonstrate the robustness, functionality and overall applicability of session types
			through an overview of the diversity of usecases in different domains
			and paradigms.

	\item	Use usecase scenarios to demonstrate tools and technologies that
			use session types in the software development process.

	\item	Share our experience in using session types for the
			implementation of different usecase scenarios.

	\item	Develop the ways and the session types terminology to
			present practical results.
\end{itemize}

\paragraph{Session Types as Protocols}
A basic 

\paragraph{Contribution}



%The entire research process for the session types discipline
%covered a broad spectrum of approaches.
%Historically, session types as a field of research was emerged
%with the definition~\cite[Honda.Vasconcelos.Kubo]
%binary session types in terms of process calculi.
%The fact that session types were defined in process calculi terms,
%which is a strict mathematical tool, enabled for the rigorous
%study of the properties of session types that followed~\cite{Gay.Hole,Yoshida.Vasconcelos}.
%The study of the properties of session types properties, in turn
%enabled a wide research on disciplines that respect session type
%principles and attempt to lift any limitations that were present.
%So we witnessed the definition of multiparty session types~\cite{Honda.Yoshida.Carbone},
%parametric session types~\cite{blah}, dynamic session types~\cite{Denieloy-Yoshida}
%eventful session types~\cite{Kouzapas.et.al}.
%In parallel with this research procedure there was the effort to provide
%experimental implementations SJ, ESJ, Pabble, 

\begin{comment}
Then there was the encoding of session types in other paradigms
Actors
Objects/Typestate
Logical interpretations/Functional
MPI



And then there was implementation of tools: SJ, LINKS, Mungo, Monitors, Scribble, Erlang

The whole research procedure was strongly motivated and driven by the need
of demonstrating that session types principles can capture different
patterns of concurrency that are applied to different actual usecases.


Following this long line of session types literature, we identify
a further need to perform a comprehensive study on the ways
session types can describe and cope with different patterns
of concurrency. In fact, today we are in a position to have
an empirical demonstration of session types.


This paper attempts to present a set of usecases from the general world
of concurrency that are expressible in session types.
This presentation achieves a two axes objective:
i) It provides with an empirical evaluation of the expressiveness of
session types inside general concurrency.
ii) Through the experience of implementation session types
can be presented in more practical terms. Different tools, paradigms, frameworks
can be presented. At the same time we identify different session type patterns
used to solve concurrency problems of different nature.



__________________________


\begin{itemize}
	\item	Two decades research.
	\item	Started with binary session types - identify the principles of session types.
	\item	The study of properties followed.
	\item	The need for using session types in general concurrency gave a strong motivation and drove
			more research.
	\item	More theoretical frameworks.
	\item	In parallel experimental implementations that use session types.
	\item	Encodings of session types -> other well studied disciplines
	\item	More implementations/frameworks
	\item	Usecases
	\item	One is presenting that usecases from general concurrency can indeed be expressed
	\item	Two is presenting session types tools and techniques and how they can be used to express usecases.

	\item	Session types and software engineering
\end{itemize}
\end{comment}
