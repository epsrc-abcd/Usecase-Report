% !TeX root = main.tex

\section{Introduction}

Session types are formal descriptions of communication protocols.
By incorporating session types into programming languages and tools,
the technology of typechecking can be used to verify the dynamic
structure of communication.
% as well as the static structure of data.
Since the introduction of session types by Honda et al.\
\cite{honda.vasconcelos.kubo:language-primitives}
more than twenty years ago, there has been a broad research effort to study
their properties and apply their principles to a range of computational models
and programming paradigms.
Current research is converging towards the development of tools and technologies
that use session types as part of the software development process.
In order to take full advantage of the possibilities of session types,
it is important to start a programme of technology transfer so that they can
be integrated into broader approaches to software engineering.

Research on session types is mainly driven by the need to demonstrate that realistic,
general patterns of
%concurrency and 
communication can be described.
A typical research paper identifies a problem in terms of a real-world use-case and 
proposes a %session-typed
framework, expressed in a strict mathematical form, 
that applies the principles of session types to the solution of the problem.
The development of session-typed tools and technologies is a consequence of this
research process: tools are developed following the theory and are applied to real-world use-cases.

This approach to presenting results about session types,
although successful within the session types community,
presents difficulties for the process of knowledge exchange
between experts and interested practitioners.

%There are numerous formal approaches (\mybf{for example?}) proposed for describing communication protocols,
%in software development.
%Because of their ability to define clear and verifiable solutions, formal
%method for the description of communication are implemented within programming languages. (\mybf{What does this mean? Are we saying that programming languages already include formal methods for the description of communication?})
%Some programming languages are even standardised for specifying telecommunication systems,
%e.g.~the set of languages standardised for telecommunication systems within the
%ISO ITU-T Z.100 series.
%In the telecommunication (\mybf{complex system}) domain, many of the requirements
%usually limit the use of formal methods.
%Furthermore, there is an unexplored area on how to balance the use of formal methods
%and the complexity in industrial software systems. (\mybf{SG: I think we need to clarify what this paragraph is saying.})
%Research in session types exhibits some promising results that can
%help explore and fill in this gap.
%In addition, the problem of weak adoption of formal methods
%by the industry has been extensively studied.
%The main barriers recognised can be found in the lack
%of education, tool support and the cost of formal methods~\cite{Davis2013}.
%But even if these barriers are removed, there comes the importance of a 
%well defined improvement strategy, that managers within a software
%organisation have to follow while adopting formal methods~\cite{Ponsard2013}.
%It is therefore necessary for a wider engagement of software engineers 
%with formal method and formal method experts.

%The above standard of presenting results on session types,
%although successful within the session types community,
%it presents difficulties in the knowledge exchange process between 
%session types experts and someone, not necessarily academic, interested
%in session types. The difficulties are summarised in the points
%below:
%
\begin{enumerate}
	\item	There is a huge volume of research information about session types.
			Interested individuals need to invest substantial effort to find
			and study a large and diverse literature.

	\item	There is no uniform approach or terminology
			in the presentation of research results.
			Due to the diversity of session types,
			there are many terms that derive from different
			disciplines and are extended to refer to
			the same concepts within session types.
			This may lead to confusion among less
			expert readers.
			%An example is the use of the terms \emph{typestate},
			%\emph{session type} and \emph{protocol} for essentially the same concept
			%in the context of object-oriented programming.

	\item	%Individual
			Results about session types are often presented in
			isolation and not placed in the context of the whole field.
			%Obtaining a
			Broader understanding of the subject requires a systematic
			literature search, which may be off-putting for non-experts.

%	\item	Session type results are often presented in
%			isolation with respect to other results.
%			A researcher will have to perform research
%			following a series of publications and
%			technical reports in order to have a broader
%			understanding of the subject.

	\item	The majority of results are tightly coupled
			with a high level of formal technicality, making the
			comprehension of session types by people without
			a formal background even more difficult.
\end{enumerate}
%
Summarising this discussion, we conclude that a wider
audience would need to study a large series of partial
results and at the same time filter out a huge amount
of unnecessary technical detail in order to achieve a satisfactory
level of understanding of session types. This situation is an obstacle for
the adoption of session types as part of practical software development.

%More generally,
The problem of weak industrial adoption of formal methods
has been extensively studied. The main barriers are lack
of information, lack of tool support, and cost~\cite{Davis2013}.
Even if these barriers are removed, there is still the need for a 
well-defined strategy for managers within a software
organisation to follow while adopting formal methods~\cite{Ponsard2013}.
It is therefore necessary to broaden the engagement of software engineers 
with formal methods and formal methods experts.

%Motivated by the goal of integrating session types into
%the software development process, this paper identifies the problem of accessibility
%of session types to a wider academic and industrial audience. 
%Our aims are:
%(i) to demonstrate session types in a comprehensive way and in
%terms that are easier to understand by non-experts; and
%(ii) to enable a process that will
%integrate session types into a broader context in computing.

Our approach to meeting these objectives is to establish a common
ground that can be used for knowledge exchange between researchers.
This paper identifies as a common means of communication the demonstration
of practical scenarios from a broad area of application in computing.
Use-cases can bridge the communication gap between researchers and can help
the wider promotion of the principles of session types.

Furthermore, current research directions in session types are
pushing towards a dialectic between session types and disciplines
such as software engineering and system design and implementation.
Evidence for this trend comes from the research experience
gained from the project
``From Data Types to Session Types: A Basis for Concurrency and Distribution''%\footnote{
%From Data Types to Session Types: A Basis for Concurrency and Distribution \url{http://groups.inf.ed.ac.uk/abcd/}}
(ABCD for short)~\cite{ABCD}.
The goals of the ABCD project include (i)
working with industrial partners towards integrating
session types in real world use-cases; and (ii)
compiling a set of use-cases
for session types that can be used for current and future research on
the design and development of frameworks and technologies.
%
%More concretely, 
The aims of this paper are to:
%
\begin{enumerate}
	\item	Describe in non-technical language, and develop common terminology
			for, the mathematical terms currently used in the theory of
			session types (\secref{sessions_software} and \secref{sessions_scribble}).
%			The terminology is then used to describe the practical use-cases
%			that are exhibited in the paper.

	\item	Demonstrate the methods that are currently used to present,
			analyse, and express an application in terms of session types
			(\secref{sessions_integrate}).
			The paper includes a discussion on their possible adoption by
			software engineers in the software development process
			(\secref{sessions_integrate} and \secref{session_engineeering}).

%			These methods include the Scribble~\cite{scribble}
%			protocol description language and tool chain and different
%			diagrammatic languages. \sg{But are we really presenting these diagrams as part of the session types approach?}

	\item	Demonstrate the robustness, functionality, and overall applicability
			of session types through a diverse overview of use-cases (\secref{usecases})
			that cover:
			(i) a range of domains that exhibit different computational needs;
			(ii) interpretations of session types in several
			programming paradigms; and
			(iii) current tools and technologies that integrate session types
			into the software development process.
	%
%			\begin{enumerate}[label=$\bullet$]
%
%				\item	Different domains that exhibit different computational needs.
%
%				\item	Different interpretations of session types
%						depending on the underlying programming paradigm used to implement a use-case.
%						This would help realise the perspective of applying session types to
%						further programming paradigms.
%
%						\dk{Communication based programming, events, typestate-OO, functional
%						programming}
%
%				\item	Current tools and
%						technologies that use session types in the software development
%						process.
%			\end{enumerate}

%	\item	Demonstrate software design and development techniques that
%			are derived from the principles of session types. Engineers and
%			developers should be able to appreciate the usage of session types
%			in the software development process.
		
%	\item	Share the experience of using session types for the
%			implementation of different use-case scenarios. This
%			is the first paper to exhibit diverse practical results
%			and can be used as a reference point for future work.
\end{enumerate}


%\begin{enumerate}
%	\item	Describe in non-technical language, and develop common terminology
%			for, the mathematical terms currently used in the theory of
%			session types.
%			The terminology is then used to describe the practical use-cases
%			that are exhibited in the paper.
%
%	\item	Demonstrate the methods that are currently used to present,
%			analyse, and express an application in terms of session types.
%			These methods include the Scribble~\cite{scribble}
%			protocol description language and tool chain and different
%			diagrammatic languages. \sg{But are we really presenting these diagrams as part of the session types approach?}
%
%	\item	Demonstrate the robustness, functionality, and overall applicability
%			of session types through a diverse overview of use-cases that cover:
%	%
%			\begin{enumerate}[label=$\bullet$]
%
%				\item	Different domains that exhibit different computational needs.
%
%				\item	Different interpretations of session types
%						depending on the underlying programming paradigm used to implement a use-case.
%						This would help realise the perspective of applying session types to
%						further programming paradigms.
%
%						\dk{Communication based programming, events, typestate-OO, functional
%						programming}
%
%				\item	Current tools and
%						technologies that use session types in the software development
%						process.
%			\end{enumerate}
%
%	\item	Demonstrate software design and development techniques that
%			are derived from the principles of session types. Engineers and
%			developers should be able to appreciate the usage of session types
%			in the software development process.
%		
%	\item	Share the experience of using session types for the
%			implementation of different use-case scenarios. This
%			is the first paper to exhibit diverse practical results
%			and can be used as a reference point for future work.
%\end{enumerate}


This paper is intended to be used by experts and non-experts
to exchange knowledge about the overall discipline of session types,
and to enable future work on the application of session types.


\begin{comment}
\subsection{Overview of the paper}

The paper begins in \secref{preliminaries}
with the definition of the basic terms and notions
that are used throughout the paper for the exhibition of the usecase
results.

The usecases that are presented in the paper are compiled by the researchers
working on the ABCD project and can be found in the ABCD online repository~\cite{usecase_repository}.
In \secref{usecases} there is a summary presentation
of the usecases in the online repository that gives an overview
for each usecase and discusses its design and implementation specifics
as well as the features that are being demonstrated.

In \secref{presentation} there is the selection of two
usecases from the online repository in order to demonstrate
some of the basic notions of the application of session types.
The first example is an example of an online book-store,
where two buyers want to share the expenses to buy a book.
The online book-store is consider in the bibliography as the
standard example for the presentation session types.
It first demonstrates the basic send receive operations
between multiple entities in a concurrent protocol. A second
feature is the fact that it demonstrate in a comprehensive
way the interaction logic between the entities.
The implementation of the book-store usecase using different
technologies gives a further insight to the applicability of
session types.
The second usecase presented in full is the design and implementation
of the Simple Mail Transfer Protocol~\cite{citation_needed} (SMTP),
as a way to demonstrate how session types cope with real network
protocols. SMTP is a protocol based on states. The increased complexity
of states makes the design and implementation of the SMTP a demanding
procedure. The paper claims that the use of session types can reduce
the demands of the implementation of SMTP.

\secref{engineering} makes a summary of the software engineering
practices that are identified... \dk{put more}
\end{comment}


\begin{comment}

Session types exhibit robustness in a diverse universe of applications.
We start from the rigorous analysis techniques that are developed theoretically,
and ensure a number of desirable properties in concurrent programs and we go
to the interpretation of session types into different programming paradigms
and to the usage of session types as design techniques.
The way session types can be presented plays an important role in their
overall understanding by different groups of people and the information
community in general. Below we give some basic perspective on what
is a session type.

\paragraph{Session types as protocols}
The simplest view on what is a session type is the view of a
structured protocol, that describes the communication between
a system of multiple participants.
The structure of a protocol offers an understanding of the
communication specification of the entire system.
In this paper all of the usecases we present are describe
in terms of a structure protocol using Scribble~\cite{scribble}
as the protocol language. We believe that Scribble plays a
central role in the applicability of session types, due
to the fact that it bridges the gap between a communication
specification and the implementation of the specification.


\paragraph{Session types as a software design technique}
A session type can also be seen as a technique for designing
software, no matter if this is a top down or a bottom up approach.


\paragraph{Session types as an software analysis technique}
Another approach to session types is to be used for the analysis
of properties inside a module of software.

\end{comment}

%The entire research process for the session types discipline
%covered a broad spectrum of approaches.
%Historically, session types as a field of research was emerged
%with the definition~\cite[Honda.Vasconcelos.Kubo]
%binary session types in terms of process calculi.
%The fact that session types were defined in process calculi terms,
%which is a strict mathematical tool, enabled for the rigorous
%study of the properties of session types that followed~\cite{Gay.Hole,Yoshida.Vasconcelos}.
%The study of the properties of session types properties, in turn
%enabled a wide research on disciplines that respect session type
%principles and attempt to lift any limitations that were present.
%So we witnessed the definition of multiparty session types~\cite{Honda.Yoshida.Carbone},
%parametric session types~\cite{blah}, dynamic session types~\cite{Denieloy-Yoshida}
%eventful session types~\cite{Kouzapas.et.al}.
%In parallel with this research procedure there was the effort to provide
%experimental implementations SJ, ESJ, Pabble, 

\begin{comment}
Then there was the encoding of session types in other paradigms
Actors
Objects/Typestate
Logical interpretations/Functional
MPI



And then there was implementation of tools: SJ, LINKS, Mungo, Monitors, Scribble, Erlang

The whole research procedure was strongly motivated and driven by the need
of demonstrating that session types principles can capture different
patterns of concurrency that are applied to different actual usecases.


Following this long line of session types literature, we identify
a further need to perform a comprehensive study on the ways
session types can describe and cope with different patterns
of concurrency. In fact, today we are in a position to have
an empirical demonstration of session types.


This paper attempts to present a set of usecases from the general world
of concurrency that are expressible in session types.
This presentation achieves a two axes objective:
i) It provides with an empirical evaluation of the expressiveness of
session types inside general concurrency.
ii) Through the experience of implementation session types
can be presented in more practical terms. Different tools, paradigms, frameworks
can be presented. At the same time we identify different session type patterns
used to solve concurrency problems of different nature.



__________________________


\begin{itemize}
	\item	Two decades research.
	\item	Started with binary session types - identify the principles of session types.
	\item	The study of properties followed.
	\item	The need for using session types in general concurrency gave a strong motivation and drove
			more research.
	\item	More theoretical frameworks.
	\item	In parallel experimental implementations that use session types.
	\item	Encodings of session types -> other well studied disciplines
	\item	More implementations/frameworks
	\item	Usecases
	\item	One is presenting that usecases from general concurrency can indeed be expressed
	\item	Two is presenting session types tools and techniques and how they can be used to express usecases.

	\item	Session types and software engineering
\end{itemize}
\end{comment}
