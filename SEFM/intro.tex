\section{Introduction}

The discipline of session types was emerged during the last
two decades, as a framework for enforcing a series of
{\em good} properties in concurrent implementations of programs.
Since their original inception in the
the seminal paper by Honda et.~al.~\cite{Honda.Vasconcelos.Kubo},
there was a broad research effort to study the properties of
session types and at the same time apply session types to different
frameworks and paradigms. This subsequently lead to the development
of tools and technologies that use session types as part of the 
software development process.
This research procedure was strongly motivated and driven by the
need to demonstrate that session types principles can be used
to capture different patterns in general concurrency.
On the other hand, every work on its own is presented in isolation
in the sense that it deals with specific problems, making the world
of session types a series of partial results.

In this paper we draw motivation from the need to present a comprehensive
survey of session types, using the means of demonstration. A broad spectrum
of technologies are currently used to research on session types and as
a result there are many implementations of different usecases available
to the research community.
As part of the ABCD\footnote{From Data Types to Session Types: A Basis for Concurrency and Distribution}
we have compiled a representative set of usecases in session type research
and practice.
The first point we aim at is to use session types to overview a diversity
of usecases in different domains.
A second aim is to use scenarios in order to demonstrate the different
frameworks and paradigms and different technologies that use
session types to develop software. Furthermore, we aim to show implementation
techniques that arise when session types are applied to particular domains.

Another characteristic of the current line of session type research is
the fact that research so far is tightly coupled with formal terminology
that is less accessible to a wider research community. We believe that
now is the right time to lift the session types terminology to a terminology
used in the software and system engineering discipline.

\paragraph{Session Types as Protocols/Scribble}

\paragraph{Contribution}



%The entire research process for the session types discipline
%covered a broad spectrum of approaches.
%Historically, session types as a field of research was emerged
%with the definition~\cite[Honda.Vasconcelos.Kubo]
%binary session types in terms of process calculi.
%The fact that session types were defined in process calculi terms,
%which is a strict mathematical tool, enabled for the rigorous
%study of the properties of session types that followed~\cite{Gay.Hole,Yoshida.Vasconcelos}.
%The study of the properties of session types properties, in turn
%enabled a wide research on disciplines that respect session type
%principles and attempt to lift any limitations that were present.
%So we witnessed the definition of multiparty session types~\cite{Honda.Yoshida.Carbone},
%parametric session types~\cite{blah}, dynamic session types~\cite{Denieloy-Yoshida}
%eventful session types~\cite{Kouzapas.et.al}.
%In parallel with this research procedure there was the effort to provide
%experimental implementations SJ, ESJ, Pabble, 

\begin{comment}
Then there was the encoding of session types in other paradigms
Actors
Objects/Typestate
Logical interpretations/Functional
MPI



And then there was implementation of tools: SJ, LINKS, Mungo, Monitors, Scribble, Erlang

The whole research procedure was strongly motivated and driven by the need
of demonstrating that session types principles can capture different
patterns of concurrency that are applied to different actual usecases.


Following this long line of session types literature, we identify
a further need to perform a comprehensive study on the ways
session types can describe and cope with different patterns
of concurrency. In fact, today we are in a position to have
an empirical demonstration of session types.


This paper attempts to present a set of usecases from the general world
of concurrency that are expressible in session types.
This presentation achieves a two axes objective:
i) It provides with an empirical evaluation of the expressiveness of
session types inside general concurrency.
ii) Through the experience of implementation session types
can be presented in more practical terms. Different tools, paradigms, frameworks
can be presented. At the same time we identify different session type patterns
used to solve concurrency problems of different nature.



__________________________


\begin{itemize}
	\item	Two decades research.
	\item	Started with binary session types - identify the principles of session types.
	\item	The study of properties followed.
	\item	The need for using session types in general concurrency gave a strong motivation and drove
			more research.
	\item	More theoretical frameworks.
	\item	In parallel experimental implementations that use session types.
	\item	Encodings of session types -> other well studied disciplines
	\item	More implementations/frameworks
	\item	Usecases
	\item	One is presenting that usecases from general concurrency can indeed be expressed
	\item	Two is presenting session types tools and techniques and how they can be used to express usecases.

	\item	Session types and software engineering
\end{itemize}
\end{comment}
