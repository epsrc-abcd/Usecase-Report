\section{Introduction}

The session types discipline emerged during the last
two decades, as a framework for enforcing a number
of {\em good} and desirable properties in concurrent
programs.
Since their original inception in the the seminal paper by
Honda et.~al.~\cite{honda.vasconcelos.kubo:language-primitives},
there has been a broad research effort to study the properties of
session types and at the same time apply session types to different
frameworks and paradigms. This subsequently lead to the development
of tools and technologies that use session types as part of the 
software development process.

Research on session types is strongly motivated and driven by
the need to demonstrate that session types principles can
capture arbitrary patterns in concurrency,
i.e.~patterns that are not restricted to particular cases.
A typical work on session types, identifies a problem
in terms of a practical usecase and proposes session types
in a mathematical form, which enables a further
theoretical study of session types properties. The development
of technologies that use session types also usually rely on
theoretical results and are demonstrated through the means
of usecase scenarios.

As a consequence, someone, not necessarily academic,
interested in acquiring knowledge on the session type discipline
will face three main challenges:
%
\begin{enumerate}%[$\bullet$]
	\item	There is a huge volume of research information on
			session types, which is associated with different
			research domains and objectives. In other words
			there is no uniform way of approaching session types.

	\item	Results are often presented in isolation with
			respect to other results from session types.
			A researcher will have to perform research
			following a series of publications and
			technical reports.

	\item	A lot of the results are tightly coupled with
			a high level of technicality making the
			comprehension of session types by non experts
			more difficult.
\end{enumerate}
%
Subsequently, a wider audience would need to study a series
of partial results and at the same time filter out a huge
amount of unnecessary details in order to understand session types
as a whole.
The three problems identified above, will not only present
difficulties to a session types newcomer, it might also
discourage his or her efforts in understanding session types.

In this paper we identify the problem of accessibility of session
types to a wider academic and industrial audience.
We draw motivation by the need to demonstrate session
types in a comprehensive way and in terms that are easier to
understand by less experts in the domain of session types.
The method we use is the method of demonstrating the practical
experience we gained in session types that is reflected by a set of
usecase scenarios~\cite{usecase_repository} that we have implemented as part of 
the ABCD project~\cite{ABCD}\footnote{From Data Types to Session Types:
A Basis for Concurrency and Distribution}.

More concretely we aim to:
%
\begin{enumerate}%[$\bullet$]
	\item	Demonstrate the robustness, functionality and overall applicability of session types
			through an overview of the diversity of usecases in different domains
			and paradigms.

	\item	Use usecase scenarios to demonstrate tools and technologies that
			use session types in the software development process.

	\item	Share our experience in using session types for the
			implementation of different usecase scenarios.

	\item	Develop the ways and the session types terminology to
			present practical results.

	\item	A paper that is used by experts as a way to communicate
			their ideas with other researcher from different computer
			science disciplines.
\end{enumerate}

\paragraph{Session Types as Protocols}
A basic 

\paragraph{Contribution}



%The entire research process for the session types discipline
%covered a broad spectrum of approaches.
%Historically, session types as a field of research was emerged
%with the definition~\cite[Honda.Vasconcelos.Kubo]
%binary session types in terms of process calculi.
%The fact that session types were defined in process calculi terms,
%which is a strict mathematical tool, enabled for the rigorous
%study of the properties of session types that followed~\cite{Gay.Hole,Yoshida.Vasconcelos}.
%The study of the properties of session types properties, in turn
%enabled a wide research on disciplines that respect session type
%principles and attempt to lift any limitations that were present.
%So we witnessed the definition of multiparty session types~\cite{Honda.Yoshida.Carbone},
%parametric session types~\cite{blah}, dynamic session types~\cite{Denieloy-Yoshida}
%eventful session types~\cite{Kouzapas.et.al}.
%In parallel with this research procedure there was the effort to provide
%experimental implementations SJ, ESJ, Pabble, 

\begin{comment}
Then there was the encoding of session types in other paradigms
Actors
Objects/Typestate
Logical interpretations/Functional
MPI



And then there was implementation of tools: SJ, LINKS, Mungo, Monitors, Scribble, Erlang

The whole research procedure was strongly motivated and driven by the need
of demonstrating that session types principles can capture different
patterns of concurrency that are applied to different actual usecases.


Following this long line of session types literature, we identify
a further need to perform a comprehensive study on the ways
session types can describe and cope with different patterns
of concurrency. In fact, today we are in a position to have
an empirical demonstration of session types.


This paper attempts to present a set of usecases from the general world
of concurrency that are expressible in session types.
This presentation achieves a two axes objective:
i) It provides with an empirical evaluation of the expressiveness of
session types inside general concurrency.
ii) Through the experience of implementation session types
can be presented in more practical terms. Different tools, paradigms, frameworks
can be presented. At the same time we identify different session type patterns
used to solve concurrency problems of different nature.



__________________________


\begin{itemize}
	\item	Two decades research.
	\item	Started with binary session types - identify the principles of session types.
	\item	The study of properties followed.
	\item	The need for using session types in general concurrency gave a strong motivation and drove
			more research.
	\item	More theoretical frameworks.
	\item	In parallel experimental implementations that use session types.
	\item	Encodings of session types -> other well studied disciplines
	\item	More implementations/frameworks
	\item	Usecases
	\item	One is presenting that usecases from general concurrency can indeed be expressed
	\item	Two is presenting session types tools and techniques and how they can be used to express usecases.

	\item	Session types and software engineering
\end{itemize}
\end{comment}
