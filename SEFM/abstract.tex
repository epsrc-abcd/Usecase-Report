% !TeX root = main.tex

\begin{abstract}
Session types are formal descriptions of communication protocols.
The formal nature of session types enables their incorporation
into programming languages and tools so they can be used
to drive the software development process.
%which can be incorporated into programming languages and tools
%so that typechecking can be used to verify the dynamic structure
%of communication as well as the static structure of data.
Research on session types has moved from theoretical studies
towards practical applications, and the field is maturing to the
point where session types can be integrated into software
engineering methodologies.
However, most of the literature is not sufficiently accessible to a
wider academic and industrial audience;
more systematic technology transfer is required.
The present paper provides an entry point to a substantial repository
of practical use-cases for session types in a range of application domains.
It aims to be an accessible introduction to session types as a practical
foundation for the development of communication-oriented software, and to
encourage the adoption of session-type-based tools.
%
%	Research on session types is now mature
%	enough to expand in the discipline of system and software engineering
%	and design as a part of the process to integrate
%	session types in a broader context.
%	However, this paper identifies the problem
%	of accessibility of session types to a wider audience,
%	not necessarily academic.
%	The approach used to tackle the above problems is to use the
%	experience gained from the development of practical usecases,
%	that use session types in the software development process, as
%	a common ground for knowledge transfer.
%	The robustness, functionality, and overall applicability of
%	session types is shown through a diverse overview of
%	usecases in different domains, implemented using different
%	technologies.
\end{abstract}
