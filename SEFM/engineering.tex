\section{Session types and Software Engineering}


\mybf{Write here about the main parts of software life-cycle (requirements, specification, design, implementation, verification, testing, maintenance), and software management and organisation.
	
Support the text later with the discussion of section 2 and the results of Section 3.}


Session types is a formal method that belongs to the class of type systems.
On the one hand, types is a static analysis technique applied directly
on the source code in order to verify a set of predefined properties.
In the case of session types these properties derive
from the fundamentals of {\em good} communication.
On the other hand, types are subject to a syntax,
and possibly a semantics, that describes an implementation structure.
These two facts allow for the integration of session types both in-depth
and in-breadth in the software development process. Furthermore,
this integration is supported by automated procedures.
Consequently, session types can drive the entire process of
software development. The latter claim is supported by the
experience gained from the repository presented in \secref{usecases}.

\mybf{Hint how session types can be used in the software life-cycle. e.g.
	in a top-down approach:
	specify communication requirements and build prototypes,
	use session types to describe software structure,
	type-checker/code generators/monitors can be used in implementation.
	bottom-up approach:
	design structures,
	infer specification,
	synthesis as a verification technique, etc...
	}

In contrast other formal methods, have restricted applications
in specific software development phases.
For example, model checking
narrows down to verification/testing phase of software development.
It requires specific expertise, to define
of a complete model that approximates software and furthermore,
perform semantic analysis based on non-predefined properties.
Often in the case of error detection, results are hard to analyse
and developers need to re-iterate through previous phases
to fix errors.
Another example is the usage of formal specification languages
that do not find any usage after the specification phase.


%\begin{itemize}
%	\item	Session types protocols. Structures that describe
%			a behaviour on the communication interface.
%
%	\item	Top down design: The global type to local types via projection procedure as
%			\begin{itemize}
%				\item	a way for engineers to design
%				\item	a way to co-ordinate software developers.
%			\end{itemize}
%
%	\item	Bottom up design: The inference/synthesis procedure as
%			a way for more agile independent teams to co-ordinate and
%			exchange the behaviour of interfaces as session type
%			specifications.
%
%	\item	Testing: Session typed modules can be used to test
%			a software in the development phase.
%
%	\item	Reverse engineering: inference/synthesis
%
%	\item	The relation of session types with diagrammatic languages
%			\begin{itemize}
%				\item	BPMN \cite{citation_needed}
%				\item	global graphs\cite{citation_needed}
%				\item	petri nets\cite{citation_needed}
%				\item	state machines \cite{citation_needed}
%			\end{itemize}
%\end{itemize}
