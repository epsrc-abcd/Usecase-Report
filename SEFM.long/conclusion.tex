\section{Conclusion}
\label{sec:conclusion}

This paper draws motivation from the need
of integrating session types in the software
development process. The main problem towards
this direction is the difficulty of accessing
session types, in order to initiate a process
of knowledge exchange between session types
experts and a wider academic and industrial
audience. This paper, uses as common ground
the experience gained by implementing
the on-line session types use-case
repository~\cite{usecase_repository}, developed for the
requirements of the ABCD project~\cite{ABCD},
to bridge the above gap. For the purposes of
the presentation of the repository the session terminology
and method is explained in using non-mathematical
definitions.

The ABCD repository contains a 
diverse set of use-cases, in terms of domains,
programming paradigms, and technologies.
In the domain of web-services session types
demonstrate how to specify and implement
the interaction logic between network components.
Session types can be used to implement internet
application protocols, where a session protocol
is used to specify a standard RFC. Network topologies
require a definition which is parametric on the number
of participants. Classic Concurrency problem demonstrate
how session types can offer solutions to problems
that control access to resources. The final domain
of data structures and parallel algorithms demonstrates
solutions on data structure clients and furthermore,
how session types can cope with the underlying communication
requirements of parallel algorithms.

The formal and structured nature of session types
can offer a well studied method for applying 
session types in the software development method.
The Scribble platform is a tool based on the
theory of session types and it is used as
a core tool 
for the implementation of the uses-cases in the
ABCD repository.

The main conclusions derived in this paper are: the feasibility,
the robustness and the diversity of session type solutions.
The integration of session types in the software development
process goes beyond the theory and can now be presented
in practical terms.

Furthermore, a wide opportunity of future research arises
in the wider area of applying session types in real software.
The immediate objective is to continue developing technology,
and enriching the repository with a more diverse set of
use-cases.
Furthermore, a new area of research may arise that has to
do with: i) measure the cost/effectiveness of the
application of session types in
the software development process, in order to provide
more accurate ; ii) research session type solutions for
further needs of in the software development area, especially
in the areas that need formal support in order to reduce
complexity and effort.
